Barley is one of the most produced cereal grains. Food industry and customers recently showed increasing interest to use barley as healthy food ingredients. 
However, its health beneficial effects has not been clarified yet due to lack of subjective measurement of food exposure. 
An UPLC-MS based untargeted metabolomics study was performed to discover biomarkers of barley intake in urine from a randomized cross-over study in 14 healthy volunteers having consumed whole grain barley bread and whole grain wheat bread.
6 candidate markers were proposed and putatively identified: 
two of them were reported with molecular weight and possible formulas. Other two were putatively identified as glucoronate conjugates. One was putatively identified as polyphenol or phenol metabolite. The rest one was identified as phytosterol metabolite.
Taken together, promising candidate biomarkers of barley intake appear to be phytochemicals and their metabolites. 
For cereal products, unique phytochemical metabolome pattern could indicate their intakes.
