An \acrshort{lc/ms} metabolomics study was perfomred to discover biomarkers for barley intake. 
\acrshort{lc/ms} was used to analyze whole grain barley (bran part and endosperm part separately). 
Further, \acrshort{lc/ms} was used to profile metabolome of urine from cross-over intervention. Multivariable data analysis (\acrshort{pca} and \acrshort{plsda}) was used to control data quality and select significant variables that can distinguish barley and wheat intake.

\begin{figure}[H]
    \centering
    \includegraphics[scale=0.42]{images/idsummarybarley.PNG}
    \caption{Identification summary}
    \label{fig:idsummary}
\end{figure}


Further selected potential biomarkers were identified by \acrshort{ms/ms} and data-base searching. In the end, we proposed 6 potential biomarkers and putatively identified them for barley intake from urine shown in Figure \ref{fig:idsummary}. Within these 6 markers:
\begin{itemize}
    \item Two potential markers (m/z 231.0870, 775.3401), we can only report their molecular weight and possible molecular formulas.
    \item One marker was putatively identified as polyphenol or phenol metabolite.
    \item Two markers were putatively identified as glucoronate conjugates. Their unconjugated moleculars were not clear yet.
    \item One marker was putatively identified as sterol metabolite.
\end{itemize}

These markers should be further investigated and validated to confirm the structure.

These candidate biomarkers of barley intake appeared to be phytochemicals and their metabolites. 
Phytochemicals are chemical compounds produced by plants to help them thrive or thwart competitors, predators, or pathogens.
As a term, phytochemicals is generally used to describe plant compounds that are under research with unestablished effects on health and are not scientifically defined as essential nutrients.

In whole grain cereals, alkylresorcinols, benzoxazinoids, flavonoids, lignans and phytosterols were reported\cite{Koistinen2017}. From the perspective of systems biology, these phytochemicals are metabolites of plants controlled by gene expression. Therefore, different cereals, or more generally different plant-source food, should have their unique phytochemical profile. 

While human consumes the plant-source food, human's metabolome from biofluids  should also demonstrate unique pattern that can be tracked back to these phytochemicals. Therefore, unique metabolome pattern of phytochemical could differentiate and indicate plant-source food intakes. 

However, food is very complex biological and chemical mixtures. Phytochemicals only make up its small proportion. Identification, quantification methods and databases have not been established yet. Discovery of biomarkers for these plant-based phytochemicals could be progressed with more insights into these phytochemicals, but vice versa.