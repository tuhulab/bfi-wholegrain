\subsection{Barley}
\subsubsection{Barley in the farm}
Historically, barley (\textit{Hordeum vulgare}) has been an important cereal grain. It was among the first domesticated plants during the hundreds or thousands of years. It played an important role for human's transition from a hunting and gathering to agrarian lifestyle\cite{Baik2008,barleybook}. 

Nowadays, barley is one of the most produced grains in the farm. In 2016, 143 million tonnes barley were produced, ranked as 4th most produced grains, behind maize, rice and wheat. European Union countries produced 63\% barley of the world. Denmark produced 3.9 million tonnes barley in 2016 which was ranked as 11th most produced country of barley\cite{FAO2016}.

From the perspective of plant science, barley is arguably the most widely adapted cereal grain species with good drought, cold, and salt tolerance\cite{Ullrich2010}.
It is cultivated both in highly productive agricultural systems and also in marginal and subsistence environments\cite{barleyfeedtheworld}.
Barley's ability to adapt to multiple biotic and abiotic stresses\cite{barleyfeedtheworld} will be crucial to its future exploitation to deal with global food security issue.

\subsubsection{Barley on the table}
% table 
The status of barley on the table is incompatible with its large production 
in the farm. Barley for direct food use only remains important in few areas, e.g. Asia and northern Africa. Barley has been majorly used as a feed, malting and brewing grain\cite{barleyfeedtheworld,Baik2008}. About two-thirds were used for feed, one-third for malting and only about 2\% of total production were used for food directly\cite{Baik2008}.

Wheat and rice surpassed barley on the table, partially because wheat and rice products give a better mouth-feel and quality\cite{Baik2008}. Other reasons include relatively few efforts or attempts have been devoted to systematically breed and develop barley varieties for food uses, as well as food processing technique and product development\cite{Baik2008}. By far, the quality standard of barley for food use has not been well established, making it difficult for food manufacturers to select raw materials suitable for use in specific food products\cite{Baik2008}.

\subsubsection{Health Benefits of Whole Grain Barley}
Currently, research results of whole grain barley's health beneficial effects are still controversial. 

Some positive research results attributed barley's health benefit effects to its beta-glucan contents or rich phytochemicals. Barley contains 4.6\% beta-glucan by dry weight, while wheat, rye and oats contains 0.8\%, 1.5\% and 5.0\% respectively\cite{frolich2013whole}. Cereal beta-glucan was claimed to be capable of reducing blood serum cholesterol and regulating blood glucose levels\cite{frolich2013whole}. Some epidemiological studies also showed whole-grain cereals can protect against obesity, diabetes, \acrshort{cvd} and cancers\cite{Fardet2010}. Besides fibers of cereal, new hypothesis was also proposed that polyphenols, sulfur compounds, lignin and phytic acid, their antioxidant activities also contribute to barley's health benefit effects\cite{fardet2010}.

However, research results from University of Copenhagen showed that although beta-glucan of barley can induce satiety and reduce energy intake\cite{barleysatiety}, it did not affect cholesterol metabolism\cite{Ibrugger2013}. Barley's health beneficial effects and associated metabolic pathways still need further research results to achieve a clear conclusion.

A major problem underlying estimating barley's health benefit effects is lacking an objective measurement of its exposure. It will be further discussed in later text.
%Whole-grain wheat is also a rich source of methyl donors and lipotropes (methionine, betaine, choline, inositol and folates) that may be involved in cardiovascular and/or hepatic protection, lipid metabolism and DNA methylation\cite{fardet2010}.

%Potential protective effects of bound phenolic acids within the colon, of the B-complex vitamins on the nervous system and mental health, of oligosaccharides as prebiotics, of compounds associated with skeleton health, and of other compounds such as a-linolenic acid, policosanol, melatonin, phytosterols and para-aminobenzoic acid also deserve to be studied in more depth.\cite{fardet2010}

%High intake of whole grain cereal may benefit health in many aspects. This hypothesis has been under debate since long ago. Investigating relationship between specific food exposure and health status is important for nutritional study. However, measuring food exposure is normally based on subjects self-report1. It intrinsically consists bias and error. A potential way to avoid bias and error is by measuring food intake biomarker in subjects biological fluids1. 
%Biomarkers for whole grain intake has been widely used as a method to assess the intake. Alkylresorcinol in plasma and urine has been proved as intake biomarker for whole grain wheat and rye2. However, barley, the 4th most produced cereal in the world lacks the discovery of intake biomarker.
%Previous clinical research tested whole grain barley intervention on cardiovascular disease (CVD) as a comparison with whole grain wheat. However, no significant evidence can prove whole grain barley is able to relief CVD risk factor.

\subsubsection{Renewd Interest of Barley in Food Industry}
In the past decades, barley has attracted a lot of interest of food producers and consumers to be used as a healthy food ingredient\cite{Baik2008}, especially in baking industry\cite{Sullivan2013}.

Renewed interest of barley for food uses largely centres around the effects of beta-glucans on lowering blood cholesterol levels and glycemic index. In addition, whole grain barley foods were considered to be capable of increasing satiety, reducing energy intake and weight loss\cite{Baik2008}.

There is a great potential for food industry to develop barley-based healthy food as a substitute for popularly used cereal grains such as wheat, oat, rice, and maize\cite{Baik2008}.

\subsection{Whole Grain Cereal Intake Biomarkers}
\subsubsection{\acrfull{bfis}}
The research interests and tasks of modern nutrition science have shifted gradually these years \cite{Scalbert2014}. Especially in developed or newly-developed countries, e.g. Denmark and China, nutrition associated diseases trends from essential nutrients deficiency disease to chronic diseases, e.g. obesity, diabetes, \acrshort{cvd} and cancer. The research objects also changed from 7 types of essential nutrients to a large group of non-essential nutrients. It attracts increasing interest to research non-essential nutrients' health benefit effects, chronic disease prevention effects, their metabolic and signaling pathways etc.

This trend in nutrition science reseach also raised a question, 'how to accurately, objectively and economically measure the intake of these non-essential nutrients?' By far, the application of \acrfull{bfis} has been proven that it can provide a more objective method to measure dietary exposures with a high precision and detail\cite{Scalbert2014}.

Dietary exposure has traditionally been measured by self-reported methods, e.g. 24-hour dietary recalls or food-frequency questionnaires\cite{Rutishauser2005DietaryMeasurements.}. 
However, these methods consist a lot of subjective factors, such as recall bias and difficulty in assessing portion sizes\cite{Scalbert2014}. 
Additionally, such as cigarettes and alcohol consumption could also be biased by self-reporting methods because of cultural and societal attitudes towards these consumptions. 
Hence, these self-reporting methods intrinsically consist bias and errors.

The wrongly-estimated dietary exposure will further affect the study results of correlations between dietary exposures and disease frequencies, drawing controversial or even wrong conclusions\cite{Scalbert2014}.

\subsubsection{Alkylresorcinols}
Alkylresorcinols and their metabolites can be used as whole grain cereal intake biomarkers.
It was reported and validated in around a dozen of researches both in intervention study and population study, shown in the table.
Alkylresorcinols and their metabolites can be detected both in urinary and plasma samples.

However, vast majority of these studies mentioned above were focused on whole grain wheat or rye, or taking into account of whole grain cereals group. By far, there exists no biomarker specifically to indicate barley intakes.

%\newgeometry{landscape}

%\begin{landscape}
\begin{center}
\scalebox{0.6}
{
\begin{small}


\begin{tabular}{|m{0.5cm}|m{1.8cm}|m{2.5cm}|m{2.5cm}|m{2.5cm}|m{2cm}|m{2cm}|m{2cm}|m{0.8cm}|m{0.3cm}}
%\begin{tabu} to 1.2\textwidth { | c | c |X[c]| X[c]| X[c]| X[c]| X[c]|c| c| c}
 \hline
No & Authors & Experimental\newline methods & Food types & Compounds & Subjects & Matrix & Reference\\ \hline
1&	Wierzbicka, R etc.&	three-day weighed food record &	Whole grain cereals	&alkylresorcinol metabolites	& 69 Swedish	& urine &	\cite{ISI:000404730400012}\\ \hline
2&	Zhu, YD etc.&Diet Intervention&Whole grain wheat&alkylresorcinol metabolites,
benzoxazinoid derivatives,phenolic acid derivatives&12 healthy participants&urine&	\cite{ISI:000387249200001}\\ \hline
3&	Garcia-Aloy, M etc.&	Self-reported food frequency questionnaires&	whole grain bread	&phytochemicals (benzoxazinoids, alkylresorcinol metabolites) &	155 subjects	&urine	&\cite{ISI:000384784900008}\\ \hline
4&	Magnusdottir, OK etc.&	controlled diet&whole grain rye&	alkylresorcinol C17:0/C21:0 ratio&	93 metabolic syndrome patients in Nordic countries&	plasma&	\cite{ISI:000343662800061}\\ \hline
5&	Lappi, J etc.&	Diet Intervention&	whole grain and fibre riched rye bread&	alkylrecorsinol& &		plasma	&\cite{Sawicki2016}\\ \hline
6&	Ma, JT etc.	&Self-reported food frequency questionnaires&	whole grain cereals&	alkylrecorsinol&	407 olders	&plasma&	\cite{ISI:000309032000011}\\ \hline
7&	Ross, AB etc.&	Diet Intervention&	whole grain food (including wheat, oats, brown basmati rice, corn, rice, barley)&alkylrecorsinol&	316 overweight and obese participants&	plasma&	\cite{ISI:000298402100026}\\ \hline
8&	Andersson, A etc.&	Food records&	whole grain wheat and rye&	alkylrecorsinol&	72 Swedish adults&	nonfasting and fasting plasma&	\cite{ISI:000294523500019}\\ \hline
9&	Landberg, R etc. &	semi-quantitative food frequency questionnaires&	rye bread&	alkylrecorsinol&	360 post-menopausal women&	plasma&	\cite{Landberg2009}\\ \hline
10&	Montonen, J. etc.&	Self-reported food frequency questionnaires&	Whole grain food&	alkylrecorsinol&	100 healthy adults&	plasma	&\cite{ISI:000279623300007}\\ \hline
11&	Guyman, LA etc.&	three-day food record and food frequency questionnaires&	Whole grain food&	3-(3,5-dihydroxyphenyl)-1-propanoic acid& &	urine&	\cite{ISI:000259554500019}\\ \hline
12&	Landberg, R etc.&	Diet Intervention&	whole grain wheat and rye&	alkylrecorsinol&	22 women and 8 men&	plasma	&\cite{ISI:000255012000007}\\ \hline
%\end{tabu}
\end{tabular}
\end{small}
}
\end{center}

\subsection{Novel Dietary Biomarkers Identification by Untargetted \\ Metabolomics}
\subsubsection{Metabolome Profiling Techniques}
With the development of high-throughout analytical technologies, 3 techniques has been majorly used in untargetted metabolomics study to identify novel dietary biomarkers. They are \acrfull{lc/ms}, \\\acrfull{gc/ms} and \acrfull{nmr} \cite{Scalbert2014}. 

The detailed comparisons of these techniques are out of this project scope. But, within these 3 techniques, \acrshort{lc/ms} has an outstanding role and is briefly described:

It has been proven with the advantages of:

(1) high sensitivities: the low detection limit allows tiny amount of metabolites detection.

(2) wide coverages of analytes: abundant commercially available columns, different mobile phase combination, gradient elution and temperature etc. can be optimized to adapt to different analytical tasks.

However, compared its omic-cousins, metabolomics is still not a global technique. Not all metabolites can be profiled in a single analysis.

\subsubsection{Metabolomics data analysis}
Metabolomics data analysis generally consists data preprocessing, data alignment, data normalization and signal correction followed by the analysis through various statistical methods \cite{Scalbert2014}

\subsubsection{Compound Identification and Validation}
Compound identification has been recognized as the most difficult aspect hurdling the discovery of novel biomarkers. Not only limited to \acrshort{bfis}, compound identification troubles almost all Metabolomics researches, including disease diagnose biomarkers etc.

In the 1st Copenhagen Clinical Metabolomics Conference held in Hellerup, Denmark on 25-26 Oct 2018, experts from diverse Metabolomics research areas proposed several insightful suggestions that currently or in the future could impact the methods for compound identifications. These orally presented proposals were summarized as following:

\begin{itemize}
    \item \textbf{Improvement of analytical instruments for metabolome profiling:} as proposed by Karolina\footnote{Karolina Sulek, Novo Nordish Foundation Center for Protein Research, University of Copenhagen, Denmark}, the signals of biomarkers could be very low and the detection can easily be interfered by noise or other compounds in metabolome profiling. 
    
    The development of novel analytical instruments could increase analytes coverage or increased resolving power which may have an impact on compound identification in the future.
    
    For example, \acrfull{tims-pasef} can profile metabolome with high resolution generating a 4-dimentional dataset (ion mobility, retention time, mass to charge ratio and intensity). The unique 4-D fingerprint can make compounds identification easier.
    
    \item \textbf{Bioinformatics, database and data sharing:} as proposed by Cristina\footnote{Cristina Legido-Quigley, Steno Diabetes Center, Denmark}, Gabi\footnote{Gabi Kastenmüller, HelmholtzZentrum München, Denmark}, David\footnote{David Wishart, University of Albertaz, Canada}, Niles\footnote{Nils Færgeman, University of Southern Denmark, Denmark} and Pieter\footnotemark, compound identification could be revolutionized by using bioinformatics such as computational chemistry to predict MS/MS spectra of metabolites. Then these data can be shared via database such as\acrfull{hmdb} \cite{hmdb} and be accessible by the whole community.
    
    Pieter\footnotemark[\value{footnote}] also pointed out a lot of spectra were generated but not open to the community. \footnotetext{Pieter Dorrestein, University of California San Diego, USA}He proposed that metabolomics researchers should share \acrshort{ms/ms} spectra in \acrfull{gnps} \cite{GNPS}. These spectra can be accessed by the whole community. In addition, a computer algorithm namely 'molecular network' can calculate the similarities between unknown compounds and annotated compounds in the database. Hence, researchers can get annotation suggestions of unknown compounds. 
    
    \item \textbf {Deepening the understanding of biological aspect of metabolism:}     Marta\footnote{Marta Cascante, University of Barcelona, Spain} proposed that, the technical aspect of metabolomics is relatively mature, such as analytical instruments and bioinformatic tools. On contray, the real bottleneck limiting deeper understanding of metabolomics is its biological aspect. If we can deepen the understanding of it biological aspects (such as signaling, metabolic pathway etc.), we can better understand metabolomics itself in return. Therefore, the compound identification could be easier.
 
    \item \textbf{'Correctly' identify the identifiable compounds:}
    Mesut\footnote{Mesut Bilgin, Danish Cancer Society, Denmark} is an expert in shotgun mass spectrometry based lipidomics.
    He proposed a unique perspective for compound identification but could be inspiring. 
    
    He pointed out that, unlike its omic counsins, identifying all metabolites is currently impossible in metabolomics research. Therefore, it is important to 'correctly' identify those identifiable compounds. 
    
     A systematic protocol for mammalian lipidome analysis has been built up by him and his colleagues\cite{Nielsen2017}. The key ideas were briefly summarized as following: 
     
     An extensive in-house database was built covering all 'biologically-feasible' lipids based on their chain length and unsaturation degree. In addition, he emphasized a strict analytical conditions. For example, the lipid extraction should always be conducted under 4 degree to avoid plastic dissolving into the organic solvents. 
     By the way, based on a lot of preliminary experiments, he chose specific brands of solvents, plastic containers, experimental temperature etc to avoid contaminants to the maximum extent. Further, he sticked strictly to these conditions. Therefore, it can reach a correct identification as well as quantification.
\end{itemize}

Besides these proposals from experts, compound identification, especially for \acrlong{bfis} discovery, identification also relies on the knowledge or progress from other disciplines, such as food chemistry, plant science. A systematic literature research is suggested to 'dig' information for compound identification\cite{Pratico2018}.

%\textbf{Hypothesis}
%Some biomarkers can indicate barley intake in urine sample based on untargeted metabolomics. 

%\textbf{Methods}
%Urine sample was analysed by LC-MS. 
%LC-MS Data will be pre-processed by MZmine through the step: 
%noise filtering, baseline correction, peak detection, deconvolution, alignment and normalization.

%After pre-processing, 
%multivariable data analysis (PCA and PLSDA) 
%by PLS toolbox will clue the potential features for later identification. In the end, potential compounds will be identified using databases as reference. MS/MS can also be used to verify the identification results.

%\textbf{Result}
%X, X, X can be whole grain barley intake biomarker.