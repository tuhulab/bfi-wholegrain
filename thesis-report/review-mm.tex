This review referred the systematic BFIRev methodology\cite{Pratico2018}. The flowchart was included in Appendix (Fig-\ref{fig:barleybiomarkerreview} and Fig-\ref{fig:wheatbiomarkerreview})

The objective of this literature review was to identify and evaluate reported potential biomarkers for dietary assessment for whole grain wheat and whole grain barley.

Keywords as suggested in the guidelines\cite{Pratico2018} were used to search in 3 database (PubMed, Web of Science, Scopus). Keywords used for searing BFI barley in human: 
(barley) AND (biomarker* OR marker* OR metabolite* OR biokinetics OR biotransformation OR pharmacokinetics) AND (intake OR meal OR diet OR ingestion OR consumption OR eating OR food) AND 
(human* OR men OR women OR patient* OR volunteer* OR participant*) AND 
(trail* or experiment OR study) AND (urine OR plasma OR blood OR serum OR excretion OR hair OR toenail OR faeces OR faecal water). The first element was changed to wheat for wheat intake biomarker searching. 

Due to limited amount of searching results, barley searching scope was expanded to animal studies. Therefore, the keyword (animal* OR goat OR sheep OR cow OR mice OR mouse* OR animal model* OR dog*) was used to replace the previous 'human*' entry. In addition, 'feed' was added to 'food' entry.

Other database including HMDB\cite{hmdb}, FoodDB\cite{foodb}, PhenolExplorer\cite{phenolexplorer}, Dictionary of Food Compound\cite{dictionary} were also used to explore compounds present exclusively in \acrshort{wg} barley and wheat. 

In order to verify the uniqueness of compound, the same keywords combinations were used but with compound name instead of 'wheat' and 'barley'.