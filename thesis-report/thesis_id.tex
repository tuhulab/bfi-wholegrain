\section{Introduction}
In this master thesis work, most of my time was spent on identifying metabolites from biospecies such as urine and plasma. This is same in all biomarker discovery studies. however, this step is very time-consuming. I summaried the identification strategies i used. This might be inspiring for other researchers.

In metabolimics pipeline, identification comes after statistics. Selected features (with retention time and mass-to-charge ratio) should be identified. 

\textbf{step one: database search}
Normally, 1st step starts from searching database. the database could be a public one or in-house one. Normally in-house database contains RT, fragments, adducts, etc. This might easily match get a level-one identification.
however, searching in a public database normally can get extensive hints, e.g. C6H10O might have a lot of hints.

\textbf{step two: use reference compound}
1st, reference compound is available. then purchase them, e.g. in my case, sitostanol did not fly in our Quad method. then, I checked another method, Linda's method to make it fly.

In another case, interesting compound could be an metabolite of the reference compound. Then, we need to some in-vitro metabolism experiments, e.g. oxidation, glucuronadation and sulfazation. If they match 

second method is, use glucuronadase or sulfatease to treat the sample and analyze them again, to check whether RT, m/z and MSMS spectra match.

Another challenge is reference compound is not always available. For example in Muyao's PhD thesis, she synthesized her own compound and made a level-1 identification for spinach intake.

\section{Materials and methods}
In order to identify researches dealing with the identification of BFIs, we carried out an extensive literature search following the BFIRev methodology\cite{Pratico2018}.

 Briefly, searches were carried out in three databases (PubMed, Scopus, and ISI Web of Knowledge) in Jun 2019. In PubMed, the search terms were (nutri- tion*[Title/Abstract]) AND (biomarker*[Title] OR marker*[Title]) AND (validation*[Title/Abstract] OR validity*[Title/Abstract] OR validate*[Title/Abstract] OR assessment*[Title/Abstract]) NOT (animal OR rat OR mouse OR mice OR pig) NOT (disease*[Title] OR risk*[- Title] OR inflammat*[Title/Abstract] OR patient*[Title]). To avoid all the studies concerned with a single bio- marker while keeping studies on validation in general, we avoided using nutrient* or food* in the search strat- egy. The fields used for the other two databases were [Article Title/Abstract/Keywords] for Scopus and [Topic] for ISI Web of Science to replace [Title/Ab- stract] for PubMed. The search was limited to papers in English language and with no restriction applied for the publication dates. The review papers discussing the de- velopment and application of biomarkers in the nutri- tion field were selected in the process outlined in Fig. 1. The first draft scheme of validation criteria was based on criteria proposed in the review papers found by this literature search. This list was revised by three rounds of commenting by co-authors as well as feedback from pre- sentations at international conferences.


\section{Discussion}
\subsection{Validation prior to Identification}
In \acrshort{bfis} discovery pipeline, validation is the last step before a biomarker can be applied. 

Howver, not all metabolites are identifiable in real world, for example, because the reference compound is not available.

in order to prioritize the identification work, there's one step suggested before identification.
That is, check whether this metabolite can also distinguish this intake in another large cohort study. Normally this is the later step of BFIs discovery study, i.e. validation of metabolites. 
However, these cohort studies normally do not share their data. Also because different analytical platform, data format was used making it difficult to compare.

\subsection{Challenges and Limitations}
\subsubsection{Open data}

\subsubsection{Bioinformatics tools}
A lot of bioinformatics tools were emerged with the development of metabolomics. However, most bioinformatics tools were developed in data pre-processing, statistics, 

Not enough bioinformatics tools were developed to facilitate the metabolite identification.