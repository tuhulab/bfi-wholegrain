\documentclass[]{letter}
\usepackage[backend=biber,style=nature,%sorting=ynt
]{biblatex}
\addbibresource{barley.bib}

\begin{document}
% If you want headings on subsequent pages,
% remove the ``%'' on the next line:
% \pagestyle{headings}

\begin{letter}
\address{}

\opening{Dear Prof. Cheng and Prof. Liu}

My name is Tu Hu, a researcher working in University of Copenhagen, meanwhile writing my master thesis in discovering whole grain intake biomarker by metabolomics, and come from China :-)

I’m so excited to read your article investigating whole grain intake in Chinese population. I have a few questions and personal opinions regarding your newly published article about DHPPA and ischemic stroke risk\cite{10.1093/ajcn/nqy323}. 


\textbf{1.	Have you ever tried to investigate AR ratio C17:0/C21:0 (a marker estimating whole grain wheat and rye intake ratio in the diet) in the plasma samples?}

Based on my knowledge, rye is rarely consumed in Chinese population. Therefore, DHPPA could majorly come from whole grain wheat intake in Chinese participants. However, I’m not quite sure whether they’re still detectable in fasting plasma. They may be excreted into urine in fasting state. If pooled urine sample was collected, you may be able to quantify them there.

If AR ratio C17:0/C21:0 shows whole grain wheat constituting major whole grain exposure in Chinese population, you may further conclude actually whole grain wheat is associated with the disease risk.
This marker is also applicable for the research published by Prof. Liu in Diabetes Care Volume 41, March 2018. 
The Swedish scientist Landsberg (https://www.chalmers.se/en/departments/bio/news/Pages/Rikard-Landberg-new-Head-of-division.aspx) had two articles may get 
published on Am J Clin Nutr 2016; 104: 88-96 (https://www.ncbi.nlm.nih.gov/pubmed/27281306) reporting rye intake is more favorable for T2D prevention, . 
Because rye is rarely consumed in Chinese population, while both rye and wheat is consumed in Scandinavian countries. Researches of whole grain in Scandinavian countries may be prone to errors and deviations due to 
Therefore, I believe, observation and intervention studies in China could better elucidate the role of whole grain wheat’s health beneficial effects.

2.	Can you actually detect intact AR molecule in your LC-MS system?

Honestly speaking, in my dataset and several other projects in my group, intact AR molecule were not detectable.

\textbf{Disclaimer:} The purpose of the correspondence aims at scientific communications. The questions, opinions and viewpoint in this letter were on the basis of my personal experience and knowledge. They may not all be supported by my supervisors and affiliations.

Looing forward to hearing from you.




\signature{Tu Hu}

\closing{Sincerely yours,}

%enclosure listing
%\encl{}
\printbibliography
\end{letter}

\end{document}

