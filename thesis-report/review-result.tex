\subsection{WG barley}
The literature search retrieved 129 records after removing duplicates. None biomarker of barley intake has been reported from either human or animal studies.
%Within them, none of the studies directly investigated WG barley intake biomarkers. 
%This could be due to limited barley consumption in population.
%Although barley is the 4th most produced cereal grains world-widely, they are mostly used for brewing or feed. 
%Approximately only 4\% is directly consumed\cite{Baik2008}.

%When the scope was expanded to animal studies, the search results still did not show any direct research about BFIs. Most of animal studies were interested in how barley feed can benefit the growth of animals or quality improvement of animal-source products\cite{ISI:000272990200002,Foster2003}.

%A 2-month intervention study\cite{DeAngelis2015} incorporated 75\% refined drum wheat and 25\% WG barley. The fecal samples showed significant change in microbiota and metabolome after intervention\cite{DeAngelis2015}. However, no specific metabolite can indicate WG barley intake.

%Alkylresorcinols and their metabolites may not indicate WG barley intake. Several observation studies\cite{ISI:000309032000011,ISI:000259554500019} investigated correlation between ARs metabolites and whole grain intake. Although these studies tried to cover more whole grain species, for example, one study\cite{ISI:000259554500019} listed 7 types of commonly consumed WGs in American populations in the \acrfull{ffq}\footnote{Dark breads, High-fiber or bran cereals, Cooked cereals and grits, Regular granola, Granola bars and cereal bars, Plain popcorn (no butter) or low-fat microwave popcorn, Buttered or gular microwave popcorn}, barley was not solely listed. Therefore, although ARs and their metabolites got good correlation with these 'Whole-grain intake'. Readers should be cautious to apply these markers to \acrshort{wg} barley intake. 
%In addition, ARs concentration in cereal barley is much lower compared with \acrshort{wg} wheat and rye, with similar concentration with refined wheat and rye flours (Table-\ref{table:ars_in_plant}). 

%\begin{table}[h!]
%		\footnotesize
%\centering
%\begin{tabular}{|c|c|c|c|c|c|}
%	\hline 
%	Cereal & \makecell{Conc. range \\in cereal}& \makecell{Conc. average \\or range \\in WG flour }& \makecell{Conc. average \\in refined \\flour} & \makecell{Main\\homologues} & \makecell{C17:C21\\ homologues\\ratio} \\ 
%	\hline 
%	Rye&360-3200& 972 &90&C17, C19, C21& 0.8-0.9 \\ 
%	\hline 
%	Wheat&761-8390& 490-710 & 36& C19, C21& 0.07-0.1 \\ 
%	\hline 
%	Barley& 55.8-98.2&NA & NA& C19, C21, C25&NA  \\ 
%	\hline 
%\end{tabular} 
%	\caption{Prensence of ARs in Cereal Grains, adapted from \cite{doi:10.1021/jf0340456,ANDERSSON2010794,BORDIGA201638}(unit: \SI{}{\micro\gram}/g dm), conc. varies due to different species and milling methods.}
%	\label{table:ars_in_plant}
%\end{table}

The term ``biomarker(s)'' mentioned in retrieved results mostly referred to barley intake's \textit{effect biomarkers} as defined by Dragsted\cite{Dragsted2017} and Gao\cite{Gao2017}, such as bowel health indicators\cite{Bird2008}, postprandial glucose and insulin response\cite{Ames2015}, lipid profiles and \acrfull{cvd} markers\cite{Marungruang2018}, etc. 
For animal studies, ``biomarker'' mostly referred to the growth of animals or quality indicators of animal-source products\cite{ISI:000272990200002,Foster2003}, which could also be regarded as \textit{effect biomarkers} of animals. 
However, within above-mentioned results, intervention studies lacked objective markers for compliance monitoring. 

\subsection{WG wheat}
\subsubsection{Overview}
The literature search retrieved 312 results after removing duplicates. Some articles were found from the references of searched results. 
The final result(Table-\textbf{\ref{table:wheat_intervention}}) included one intervention study and two observational studies.
%%% EXPLAIN WHY SO FEW RESULTS WERE INCLUDED
Surprisingly to us, very few studies investigated WG wheat intake biomarkers although WGs seem to be the hotspot in food and nutrition research. The reason was described in discussion.
\begin{table}[h!]
	\scalebox{0.65}{
		\begin{tabular}{|c|c|c|c|c|c|c|c|c|}
			\hline
			
			\makecell{Food\\items} &\makecell{No.\\subjects} & \makecell{Study\\design} &  \makecell{Sample\\type}  & \makecell{Analytical\\method}& \makecell{Candidate\\biomarker(s)} & Identifier & Reference \\ 
			\hline 
			
			WGs\footnote{This cohort studies investigated WGs conc. in different EU countries' population.} & 2845& \makecell{Observational\\(11 countries)} & \makecell{\makecell{fasting and \\non-fasting\\plasma}}  & GC-MS & AR C17:0/C21:0 & \makecell{HMDB0038530\\HMDB0031035} & \cite{ISI:000334172400017} \\ 
			\hline	
			
			\makecell{WG wheat\\WG rye} & 73 & \makecell{Observational\\(after 0.1-3 years)} & plasma & GC-MS & \makecell{Total ARs\\(C17:0,C19:0,\\C21:0,C23:0,C25:0)\\Ratio of AR\\(C17:0/C21:0)} & \makecell{HMDB0038530\\HMDB0030956\\HMDB0031035\\	HMDB0038524\\HMDB0038485} &\cite{ISI:000315978400006} \\ 
			\hline 
			
			\makecell{WG wheat\\WG rye} & 39 & \makecell{Intervention\\ (cross-over)} & plasma & GC-MS & \makecell{Ratio of AR\\C17:0/C21:0} & \makecell{HMDB0038530\\HMDB0031035} &\cite{ISI:000376712600013} \\ 
			\hline 
			
			\makecell{WG wheat\\WG rye} & 15 & \makecell{Intervention\\ (cross-over)} & \makecell{plasma\\serum enterolactone} & GC-MS & \makecell{Ratio of AR\\C17:0/C21:0} & \makecell{HMDB0038530\\HMDB0031035} &\cite{10.1093/jn/137.5.1137} \\ 
			\hline 
			%\makecell{\makecell{Healthy\\ new \\nordic\\diet\footnote{containing more rye than control group}}} & 166 & \makecell{intervention,\\ parallel,\\ randomized,\\ multi-center\\ (18/24 weeks)} & plasma & GC-MS & \makecell{ratio of AR\\C17:0/C21:0} & \cite{ISI:000333777700008} \\ 
			%\hline 
	\end{tabular} }
	\caption{Biomarkers of Wheat Intake Reported in Intervention study}
	\label{table:wheat_intervention}
\end{table}

\subsubsection{Alkylresorcinols (ARs) and Homologous Ratio C17:0/C21:0}
%%%%%%%I THINK I STILL NEED TO COVER ARS%%%%%%%%%%%%%%%
%%%%%%%YES FOR SURE.
Combining total \acrfull{ars} and AR homologous ratio C17:0/C21:0 can potentially be used as a biomarker to measure \acrshort{wg} wheat intake.

%Total \acrfull{ars} can indicate WG wheat and rye intake, meanwhile AR homologous ratio C17:0/C21:0 can indicate relative compositions of WG wheat and rye in the diet. Combining these two values can indicate absolute intake amount of \acrshort{wg} wheat.

Within commonly consumed plant-based food, alkylresorcinols present high concentration exclusively in bran part of wheat and rye. AR homologous ratio C17:0/C21:0 was first reported by cereal scientists in 2004 to distinguish \acrshort{wg} rye and wheat grains \cite{Chen2004}. In grains, rye has homologous C17:0/C21:0 ratio close to 1.0, while wheat around 0.1, durum wheat around 0.01.

Further this marker was proposed by nutritionists to distinguish \acrshort{wg} rye and wheat intake. In 2005, Linko\cite{ISI:000376712600013} first investigated this biomarker in human plasma to measure food exposure. The intervention study showed the potential of this marker AR C17:0/C21:0 to distinguish \acrshort{wg} wheat and rye in diet in healthy postmenopausal women. 
For rye-dominated diet, the ratio was 0.84 and for WG wheat-dominated diet, the ratio was around 0.53. 
Further in 2007, Linko-Parvinen validated this marker in healthy adults by an intervention study \cite{10.1093/jn/137.5.1137}.
In plasma, the value was 0.1 after WG wheat intake, 0.6 after WG rye intake. In erythrocytes, the value was 0.06 and 0.33 respectively after WG wheat and rye intake. This study also implied ARs could be transported in human plasma lipoproteins.

However, the AR homologues ratio C17:0/C21:0 was unable to differentiate \acrshort{wg} diet and refined cereal diet as reported by Landberg\cite{ISI:000255012000007}. But WG diet and refined diet can be distinguished by total ARs concentration in plasma.

EPIC\footnote{European Prospective Investigation into Cancer and Nutrition} cohort study \cite{ISI:000334172400017} further proved usefulness this marker in 2014. This observational study investigated plasma ARs and the C17:0/C21:0 ratio of subjects from 10 European countries. 
The result showed that Greek, Italian, Dutch and UK participants of whom the diet was dominated by wheat, had low C17:0/C21:0 ratio in plasma. Whereas Danish, German and Swedish subjects had high C17:0/C21:0 ratio. French and Norwegian subjects had intermediate ratio. This marker showed reverse correlation with WG wheat consumption in the population.

%%%%%%NOT CORRECT%%%%%%%%%%%
%Total alkylresorcinols and their metabolites (3,5-DHPPTA, 3,5-DHPPA, 3,5-DHBA and 3,5-DHBA glycine) were not specific to WG wheat but got good correlations with WG wheat intake in rye rarely consumed countries, such as UK and USA. In these countries, WG wheat is the only alkylresorcinol source.

%%%super important
%\acrshort{ars} are absorbed by humans and can be detected in blood plasma, erythrocytes, adipose tissue\cite{ISI:000451002000016}, and in the form of polar metab- olites in urine

%%%detection method
%AR metabolites: urine/plasma: HPLC-COULOMETRIC ELECTRODE ARRAY DETECTOR

%%%METABOLITES
%longer apparent half-lives






%\acrshort{ars} (Figure\ref{fig:structure_ars}) and their metabolites  were widely reported, validated and applied biomarkers for WG wheat and rye intake. 
%Total \acrshort{ars} were used as biomarkers for overall \acrshort{wgs} wheat and rye exposure. In order to distinguish \acrshort{wg} wheat and rye. The ratio of C17:0/C21:0 was used. \acrshort{ars}, depending on different milling methods and grain species, varies the concentration and homologues compositions (Table-\ref{table:ars_in_plant}). . The ratio was further proposed as an marker to indicate which cereal dominates in the diet: if the ratio is close to 1.0, rye dominated; close to 0.1, wheat dominates\cite{ISI:000376712600013,Landberg2009}.



%%%%%%%%%%% WHEAT INTERVENTION


%In \acrfull{wg} source dominated by wheat, total ARs got good correlation with wheat intake. e.g. UK and America. 

%Meanwhile, ARs metabolites could also be a potential marker, but may not be a good marker in mixed-ARs source countries since ARs metabolites were not specific to WG wheat. 
%ARs from WG rye could also be metabolized to same products causing confounding.
%%%%%%%%%% WHEAT OBS
%\begin{table}[h!]
%	\scalebox{0.83}{
%		\begin{tabular}{|c|c|c|c|c|c|c|}
%			%header
%			\hline 
%			\makecell{Type of\\ WG} & \makecell{No.\\subjects} & \makecell{Sample\\type}  & \makecell{Analytical\\method} & \makecell{Candidate\\biomarker(s)} & \makecell{Associated\\with} & Reference \\ 
%			\hline
%			%%EU cohort
%			WGs\footnote{This cohort studies investigated WGs conc. in different EU countries' population.} & 2845 & \makecell{\makecell{fasting and \\non-fasting\\plasma}}  & GC-MS & AR C17:0/C21:0 & FFQ & \cite{ISI:000334172400017} \\ 
%			\hline	
%			
%	\end{tabular}}
%	\caption{Biomarkers of Wheat Intake Reported in Observation study}
%	\label{table:wheat_observation}
%\end{table}
%In this study\cite{ISI:000298402100026}, total ARs, rather than the ratio (C17:0/C21:0), were reported as \acrshort{wg} wheat intake biomarkers.
%This intervention was conducted in UK. In British population, the major whole grain source is wheat. Rye was rarely consumed. Therefore, plasma ARs got a good correlation with \acrfull{wg} wheat intake.

\subsubsection{Applications in Type II Diabetes Research}
This biomarker showed its usefulness in type II diabetes study proving \acrshort{wg} rye might better benefit type II diabetes prevention compared with \acrshort{wg} wheat.

An observational study showed\cite{ISI:000430455900021}, in Chinese populations AR metabolite DHPPA was correlated with lower odds of type II diabetes and impaired glucose regulation. However, DHPPA as AR metabolite can originate from either whole grain wheat or rye. 
DHPPA can only indicate total intake of \acrshort{wg} rye and wheat. But, this marker cannot provide details of each whole grain type intake.

Other two researches better showed that, \acrshort{wg} rye could be more favourable for type II diabetes prevention. An observational study showed in a population with metabolic syndrome, plasma AR C17:0/C21:0 was associated with increased insulin sensitivity\cite{ISI:000333777700008}. 
Further, it was observed that in healthy Scandinavian populations, plasma total ARs concentration was not correlated with type II diabetes risk. However higher C17:0/C21:0 ratio (implicating more rye intake) was associated with increased insulin sensitivity\cite{ISI:000378977200013}. These results implied that, a whole grain diet dominated by rye could be favourable for type II diabetes prevention. 

In EPIC cohort study\cite{ISI:000334172400017}, there was a very interesting phenomena. Rye has higher constitutions in Danish populations' WG source (70\%) than Swedish (55\%) on average. 
However, Danish participants in EPIC cohort showed lower AR C17:0/C21:0 ratio (0.37) than Swedish participants (0.43). 
Swedish participants in EPIC cohort studies were healthy adults, while Danish participants were obese or over-weights subjects. Those participants may have different dietary habits and consume less than average wg rye. This may also imply rye could be favourable in weight control.

These studies indicated that, application of biomarkers for each type of cereal intake could provide more information to study each cereal type's health beneficial effects.

\subsubsection{Other potential markers}
Searching results also showed some \textit{Food compound intake biomarkers (FCIBs)} research as defined by Gao\cite{Gao2017} such as phenolic compounds\cite{ISI:000389134200003}, benzoxazinoids (BXOs)\cite{ISI:000394168100034,ISI:000348343300015}, phytoestrogen\cite{ISI:000384082300001}, phytosterol and lignan\cite{ISI:000387249200001} and\textit{effect markers} such as microbioal derivitives\cite{ISI:000348343300015}. These compounds do not exclusively present in WG wheat. Therefore, they can not specifically indicate \acrshort{wg} wheat intake. These results were summarized in Appendix.

However, another study\cite{ISI:000387249200001} proposed using a panel of  metabolites consisting 7 AR metabolites, 5 BXO metabolites and 5 phenolic acid derivatives to objectively assess WG wheat intake. Because concentration of these phytochemicals vary in different cereal grains. Therefore, a combination of their metabolites could potentially indicate intake of different cereals. However, this conclusion needs to be further validated in other intervention and observation studies.



