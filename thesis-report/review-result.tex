\subsection{WG barley}
The literature search got 129 records after removing duplicate records from merged 3 database search results. However, within them, none of the studies directly investigated WG barley intake biomarkers. This could be explained by limited dietary exposure of barley in population. Although barley is the 4th most produced cereal grains worldwidely. Most of them is used for brewing or feed. Approximately only 4\% is directly consumed\cite{Baik2008}.

When the scope expanded to animal studies, the search results still did not show any direct research about BFIs. Most of studies were interested in how barley feed can benefit the growth of animal or quality improvement of animal-source products\cite{ISI:000272990200002,Foster2003}.

A 2-month intervention study\cite{DeAngelis2015} incorporated 75\% refined drum wheat and 25\% WG barley. The fecal samples showed significant change in microbiota and metabolome after intervention\cite{DeAngelis2015}. However, no specific metabolite can indicate WG barley intake.

ARs and their metabolites may not indicate WG barley intake. Several observation studies\cite{ISI:000309032000011,ISI:000259554500019} investigated correlation between ARs metabolites and whole grain intake. Although these studies tried to cover more whole grain species, for example, one study\cite{ISI:000259554500019} listed 7 types of regularly consumed WGs in American populations in the \acrfull{ffq}\footnote{Dark breads, High-fiber or bran cereals, Cooked cereals and grits, Regular granola, Granola bars and cereal bars, Plain popcorn (no butter) or low-fat microwave popcorn, Buttered or gular microwave popcorn}, barley was not solely listed. Therefore, although ARs and their metabolites got good correlation with these 'Whole-grain intake'. Readers should be cautious to apply these markers to \acrshort{wg} barley intake. In addition, ARs concentration in cereal barley is much lower compared with \acrshort{wg} wheat and rye, with similar concentration with refined wheat and rye flours (Table-\ref{table:ars_in_plant}). 

\begin{table}[h!]
		\footnotesize
\centering
\begin{tabular}{|c|c|c|c|c|c|}
	\hline 
	Cereal & \makecell{Conc. range \\in cereal}& \makecell{Conc. average \\or range \\in WG flour }& \makecell{Conc. average \\in refined \\flour} & \makecell{Main\\homologues} & \makecell{C17:C21\\ homologues\\ratio} \\ 
	\hline 
	Rye&360-3200& 972 &90&C17, C19, C21& 0.8-0.9 \\ 
	\hline 
	Wheat&761-8390& 490-710 & 36& C19, C21& 0.07-0.1 \\ 
	\hline 
	Barley& 55.8-98.2&NA & NA& C19, C21, C25&NA  \\ 
	\hline 
\end{tabular} 
	\caption{Prensence of ARs in Cereal Grains, adapted from \cite{doi:10.1021/jf0340456,ANDERSSON2010794,BORDIGA201638}(unit: \SI{}{\micro\gram}/g dm), conc. varies due to different species and milling methods.}
	\label{table:ars_in_plant}
\end{table}

Most search results focused on barley's \textit{effect biomarkers} as defined by Dragsted\cite{Dragsted2017} and Gao\cite{Gao2017}, such as bowel health indicators\cite{Bird2008}, postprandial glucose and insulin response\cite{Ames2015}, lipid profiles and \acrfull{cvd} markers\cite{Marungruang2018}, etc. However, in these intervention studies, compliance monitoring lacked objective markers.

Further search results in food chemistry, cereal science and plant science showed some compounds exclusively present in barley. These could give hints for further identification. The results were summarized in Table-\ref{table:candidate_biomarker_barley}.

\begin{table}[h!]
	\centering
	\small
\scalebox{0.78}{
\begin{tabular}{|c|c|c|c|c|c|}
	\hline 
	No & \makecell{Candidate\\biomarker}& Formula & \makecell{Chemical\\group} &Presence in Food & Reference\\ 
	\hline 
	1&Hordenine& C\textsubscript{10}H\textsubscript{15}NO &alkaloid&\makecell{germinating barley,\\ beer and other plants}&\cite{Gurdeniz2016}\\ 
	\hline
	4&Hordatine A&C\textsubscript{28}H\textsubscript{38}N\textsubscript{8}O\textsubscript{5}&alkaloid&\makecell{only reported\\in barley}&\makecell{FoodDB\\(002330)}\\ 
	\hline 
	4&Hordatine B&C\textsubscript{29}H\textsubscript{40}N\textsubscript{8}O\textsubscript{5}&alkaloid&\makecell{only reported\\in barley}&\makecell{FoodDB\\(002328)}\\ 
	\hline 
	2&\makecell{Distichonic\\acid A}&C\textsubscript{10}H\textsubscript{18}N\textsubscript{2}O\textsubscript{8}&\makecell{gamma amino acids\\ and derivatives}&\makecell{only reported\\in barley}&\makecell{FoodDB\\(18164)}\\
	\hline 
	3&\makecell{Distichonic\\acid B}&C\textsubscript{10}H\textsubscript{18}N\textsubscript{2}O\textsubscript{8}&\makecell{gamma amino acids\\ and derivatives}&\makecell{only reported\\in barley}&\makecell{FoodDB\\(018165)}\\ 
	\hline 
	
	5&\makecell{14,16-Nona\\cosanedione}&C\textsubscript{29}H\textsubscript{56}O\textsubscript{2}&ketone&\makecell{only reported\\in barley}&\makecell{FoodDB\\(013891)}\\ 
	\hline 
	6&N-Norgramine&C\textsubscript{10}H\textsubscript{12}N\textsubscript{2}&indole&\makecell{only reported\\in barley}&\makecell{FoodDB\\(017815)}\\ 
	\hline 
\end{tabular} }
\caption{Candidate Biomarkers for WG barley intake}
\label{table:candidate_biomarker_barley}
\end{table}
To conclude, barley, especially \acrshort{wg} barley attracted a lot of interest due to its health beneficial effects for chronic disease. However, due to barley's limited exposure in the population, currently there's no reported biomarkers can indicate its intake. However, a lot of sparse information was reported from cereal and food chemistry could give hints to identification and validations of \acrshort{wg} barley's intake biomarkers.

\subsection{WG wheat}
The literature search got 312 results after removing duplicate records from merged results. XXX were used.

\acrshort{ars} and their metabolites (3,5-DHPPTA, 3,5-DHPPA, 3,5-DHBA and 3,5-DHBA glycine) were widely reported, validated and applied biomarkers for WG wheat and rye intake. 
Total \acrshort{ars} were used as biomarkers for overall \acrshort{wgs} wheat and rye exposure. In order to distinguish \acrshort{wg} wheat and rye. The ratio of C17:0/C21:0 was used. \acrshort{ars}, depending on different milling methods and grain species, varies the concentration and homologues compositions (Table-\ref{table:ars_in_plant}). The homologues ratio C17:0/C21:0 was firstly proposed in cereal science to distinguish \acrshort{wg} rye and wheat\cite{Chen2004}. Further, the ratio was also proposed as an biomarker to indicate which cereal dominates in the diet: if the ratio is close to 1.0, rye dominated; close to 0.1, wheat dominates\cite{ISI:000376712600013,Landberg2009}.
%%%%%%%%%%% WHEAT INTERVENTION

\begin{table}[h!]
	\scalebox{0.83}{
\begin{tabular}{|c|c|c|c|c|c|c|c|}
	\hline 
	\makecell{Dietary\\factor} & \makecell{No.\\subjects} & \makecell{Study\\design}  & \makecell{Sample\\type}  & \makecell{Analytical\\method}& \makecell{Candidate\\biomarker(s)} & Reference \\ 
	\hline 
	
	\makecell{WG wheat\\WG rye} & 39 & \makecell{intervention,\\ cross-over,\\ randomized} & plasma & GC-MS & AR C17:0/C21:0 & \cite{ISI:000376712600013} \\ 
	\hline 
	
	\makecell{\makecell{Healthy\\ new \\nordic\\diet\footnote{containing more rye than control group}}} & 166 & \makecell{intervention,\\ parallel,\\ randomized,\\ multi-center\\ (18/24 weeks)} & plasma & GC-MS & AR C17:0/C21:0 & \cite{ISI:000333777700008} \\ 
	\hline 
	
	\makecell{WGs\footnote{This study was conducted in UK. WG wheat is the main WG source in British population. Considering this, although several types of WGs were used (WG wheat, corn, oats, barley and rice), WG wheat made up around 65\% of the intervention}} & 266 & \makecell{randomized,\\ parallel-group,\\ intervention} & plasma & GC-MS & Total ARs & \cite{ISI:000298402100026} \\ 
	\hline 
\end{tabular} }
\caption{Potential Biomarkers of Wheat Intake in Intervention study}
\label{table:wheat_intervention}
\end{table}

%%%%%%%%%% WHEAT OBS

\begin{table}[h!]
	\scalebox{0.85}{
	\begin{tabular}{|c|c|c|c|c|c|c|}
		%header
		\hline 
		\makecell{Type of\\ WG} & \makecell{No.\\subjects} & \makecell{Sample\\type}  & \makecell{Analytical\\method} & \makecell{Candidate\\biomarker(s)} & \makecell{Associated\\with} & Reference \\ 
		\hline
		%1st entry - US nurse
		WGs\footnote{This study was conducted in US. WG wheat is the dominant WG source in US populations.} & 104 & \makecell{spot\\urine}  & GC-MS & \makecell{ARs metabolites\\(DHBA, DHPPA)} & FFQ & \cite{ISI:000303089700010} \\ 
		\hline
		
		%2nd entry - EU cohort
		WGs\footnote{This cohort studies investigated } & 2845 & \makecell{\makecell{fasting and \\non-fasting\\plasma}}  & GC-MS & AR C17:0/C21:0 & NA & \cite{ISI:000334172400017} \\ 
		\hline
	
		%\hline 

	\end{tabular}}
	\caption{Potential Biomarkers of Wheat Intake in Observation study}
	\label{table:wheat_observation}
\end{table}


Searching results also showed some \textit{Food compound intake biomarkers (FCIBs)} research as defined by Gao\cite{Gao2017} such as phenolic compounds\cite{ISI:000389134200003}, benzoxazinoids\cite{ISI:000394168100034} and phytoestrogen\cite{ISI:000384082300001}. These compounds also present in other food,  not specific for \acrshort{wg} wheat. These results were summarized in Appendix.

Their concentrations varied in different cereal grains. Therefore, a combination of their metabolites could potentially indicate intake of different cereals.
