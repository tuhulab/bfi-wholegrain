\subsection{WG barley}
The literature search got 129 records after removing duplicates from merged 3 database results. Within them, none of the studies directly investigated WG barley intake biomarkers. This could be due to limited barley consumption in population. Although barley is the 4th most produced cereal grains world-widely, they are mostly used for brewing or feed. Approximately only 4\% is directly consumed\cite{Baik2008}.

When the scope expanded to animal studies, the search results still did not show any direct research about BFIs. Most of animal studies were interested in how barley feed can benefit the growth of animals or quality improvement of animal-source products\cite{ISI:000272990200002,Foster2003}.

A 2-month intervention study\cite{DeAngelis2015} incorporated 75\% refined drum wheat and 25\% WG barley. The fecal samples showed significant change in microbiota and metabolome after intervention\cite{DeAngelis2015}. However, no specific metabolite can indicate WG barley intake.

ARs and their metabolites may not indicate WG barley intake. Several observation studies\cite{ISI:000309032000011,ISI:000259554500019} investigated correlation between ARs metabolites and whole grain intake. Although these studies tried to cover more whole grain species, for example, one study\cite{ISI:000259554500019} listed 7 types of commonly consumed WGs in American populations in the \acrfull{ffq}\footnote{Dark breads, High-fiber or bran cereals, Cooked cereals and grits, Regular granola, Granola bars and cereal bars, Plain popcorn (no butter) or low-fat microwave popcorn, Buttered or gular microwave popcorn}, barley was not solely listed. Therefore, although ARs and their metabolites got good correlation with these 'Whole-grain intake'. Readers should be cautious to apply these markers to \acrshort{wg} barley intake. 
In addition, ARs concentration in cereal barley is much lower compared with \acrshort{wg} wheat and rye, with similar concentration with refined wheat and rye flours (Table-\ref{table:ars_in_plant}). 

\begin{table}[h!]
		\footnotesize
\centering
\begin{tabular}{|c|c|c|c|c|c|}
	\hline 
	Cereal & \makecell{Conc. range \\in cereal}& \makecell{Conc. average \\or range \\in WG flour }& \makecell{Conc. average \\in refined \\flour} & \makecell{Main\\homologues} & \makecell{C17:C21\\ homologues\\ratio} \\ 
	\hline 
	Rye&360-3200& 972 &90&C17, C19, C21& 0.8-0.9 \\ 
	\hline 
	Wheat&761-8390& 490-710 & 36& C19, C21& 0.07-0.1 \\ 
	\hline 
	Barley& 55.8-98.2&NA & NA& C19, C21, C25&NA  \\ 
	\hline 
\end{tabular} 
	\caption{Prensence of ARs in Cereal Grains, adapted from \cite{doi:10.1021/jf0340456,ANDERSSON2010794,BORDIGA201638}(unit: \SI{}{\micro\gram}/g dm), conc. varies due to different species and milling methods.}
	\label{table:ars_in_plant}
\end{table}

Most search results focused on barley's \textit{effect biomarkers} as defined by Dragsted\cite{Dragsted2017} and Gao\cite{Gao2017}, such as bowel health indicators\cite{Bird2008}, postprandial glucose and insulin response\cite{Ames2015}, lipid profiles and \acrfull{cvd} markers\cite{Marungruang2018}, etc. However, in these intervention studies, compliance monitoring lacked objective markers.


\subsection{WG wheat}
\subsubsection{Overview}
The literature search got 312 results after removing duplicates from merged results. Some articles were found from the references of searched results. One intervention study and two observation studies were included in Table-\textbf{\ref{table:wheat_intervention}}. 

%%% EXPLAIN WHY SO FEW RESULTS WERE INCLUDED
Surprisingly very few studies investigated WG wheat intake biomarkers although WGs seem to be the hotspot area in food and nutrition research. 



\subsubsection{Alkylresorcinols (ARs) and Homologous Ratio C17:0/C21:0}
%%%%%%%I THINK I STILL NEED TO COVER ARS%%%%%%%%%%%%%%%
%%%%%%%YES FOR SURE.
Total \acrfull{ars} can indicate WG wheat and rye intake, meanwhile AR homologous ratio C17:0/C21:0 can indicate relative compositions of WG wheat and rye in the diet.

Within commonly consumed plant-based food, alkylresorcinols present high concentration exclusively in bran part of wheat and rye. AR homologous ratio C17:0/C21:0 was first reported by cereal scientists in 2004 to distinguish \acrshort{wg} rye and wheat cereal \cite{Chen2004}. In grains, rye has homologous C17:0/C21:0 ratio close to 1.0, while wheat around 0.1, durum wheat around 0.01.

\begin{table}[h!]
	\scalebox{0.65}{
		\begin{tabular}{|c|c|c|c|c|c|c|c|c|}
			\hline
			
			\makecell{Food\\items} &\makecell{No.\\subjects} & \makecell{Study\\design} &  \makecell{Sample\\type}  & \makecell{Analytical\\method}& \makecell{Candidate\\biomarker(s)} & Identifier & Reference \\ 
			\hline 
			
			WGs\footnote{This cohort studies investigated WGs conc. in different EU countries' population.} & 2845& \makecell{Observation\\(11 countries)} & \makecell{\makecell{fasting and \\non-fasting\\plasma}}  & GC-MS & AR C17:0/C21:0 & \makecell{HMDB0038530\\HMDB0031035} & \cite{ISI:000334172400017} \\ 
			\hline	
			
			\makecell{WG wheat\\WG rye} & 73 & \makecell{Observation\\(after 0.1-3 years)} & plasma & GC-MS & \makecell{Total ARs\\(C17:0,C19:0,\\C21:0,C23:0,C25:0)\\Ratio of AR\\(C17:0/C21:0)} & \makecell{HMDB0038530\\HMDB0030956\\HMDB0031035\\	HMDB0038524\\HMDB0038485} &\cite{ISI:000315978400006} \\ 
			\hline 
			
			\makecell{WG wheat\\WG rye} & 39 & \makecell{Intervention\\ (cross-over)} & plasma & GC-MS & \makecell{Ratio of AR\\C17:0/C21:0} & \makecell{HMDB0038530\\HMDB0031035} &\cite{ISI:000376712600013} \\ 
			\hline 
			
			\makecell{WG wheat\\WG rye} & 15 & \makecell{Intervention\\ (cross-over)} & \makecell{plasma\\serum enterolactone} & GC-MS & \makecell{Ratio of AR\\C17:0/C21:0} & \makecell{HMDB0038530\\HMDB0031035} &\cite{10.1093/jn/137.5.1137} \\ 
			\hline 
			%\makecell{\makecell{Healthy\\ new \\nordic\\diet\footnote{containing more rye than control group}}} & 166 & \makecell{intervention,\\ parallel,\\ randomized,\\ multi-center\\ (18/24 weeks)} & plasma & GC-MS & \makecell{ratio of AR\\C17:0/C21:0} & \cite{ISI:000333777700008} \\ 
			%\hline 
			
			
	\end{tabular} }
	\caption{Biomarkers of Wheat Intake Reported in Intervention study}
	\label{table:wheat_intervention}
\end{table}

Further this marker was proposed by nutritionists to distinguish \acrshort{wg} rye and wheat intake. In 2005, Linko\cite{ISI:000376712600013} first investigated this biomarker in human plasma to measure food exposure. The intervention study showed the potential of this marker AR C17:0/C21:0 to distinguish \acrshort{wg} wheat and rye in diet in healthy postmenopausal women. 
For rye-dominated diet, the ratio was 0.84 and for WG wheat-dominated diet, the ratio was around 0.53. 
Further in 2007, Linko-Parvinen\footnote{Probably, she married and changed last name. (This footnote will be deleted in final thesis)} validated this marker in healthy adults by an intervention study \cite{10.1093/jn/137.5.1137}.
In plasma, the value was 0.1 after WG wheat intake, 0.6 after WG rye intake. In erythrocytes, the value was 0.06, 0.33 respectively after WG wheat and rye intake. This study also implied ARs could be transported in human plasma lipoproteins.

However, the AR homologues ratio C17:0/C21:0 was unable to differentiate \acrshort{wg} diet and refined cereal diet as reported by Landsberg\cite{ISI:000255012000007}. But WG diet and refined diet can be distinguished by total ARs concentration in plasma.

EPIC\footnote{European Prospective Investigation into Cancer and Nutrition} cohort study \cite{ISI:000334172400017}further validated this marker in 2014. This observation study investigated plasma ARs and the C17:0/C21:0 ratio of subjects from 10 European countries. 
The result showed that Greek, Italian, Dutch and UK participants of whom the diet was dominated by wheat, had low C17:0/C21:0 ratio in plasma. Whereas Danish, German and Swedish subjects had high C17:0/C21:0 ratio. French and Norwegian subjects had intermediate ratio. This marker showed reverse correlation with WG wheat consumption in the population.

%%%%%%NOT CORRECT%%%%%%%%%%%
%Total alkylresorcinols and their metabolites (3,5-DHPPTA, 3,5-DHPPA, 3,5-DHBA and 3,5-DHBA glycine) were not specific to WG wheat but got good correlations with WG wheat intake in rye rarely consumed countries, such as UK and USA. In these countries, WG wheat is the only alkylresorcinol source.

%%%super important
%\acrshort{ars} are absorbed by humans and can be detected in blood plasma, erythrocytes, adipose tissue\cite{ISI:000451002000016}, and in the form of polar metab- olites in urine

%%%detection method
%AR metabolites: urine/plasma: HPLC-COULOMETRIC ELECTRODE ARRAY DETECTOR

%%%METABOLITES
%longer apparent half-lives






%\acrshort{ars} (Figure\ref{fig:structure_ars}) and their metabolites  were widely reported, validated and applied biomarkers for WG wheat and rye intake. 
%Total \acrshort{ars} were used as biomarkers for overall \acrshort{wgs} wheat and rye exposure. In order to distinguish \acrshort{wg} wheat and rye. The ratio of C17:0/C21:0 was used. \acrshort{ars}, depending on different milling methods and grain species, varies the concentration and homologues compositions (Table-\ref{table:ars_in_plant}). . The ratio was further proposed as an marker to indicate which cereal dominates in the diet: if the ratio is close to 1.0, rye dominated; close to 0.1, wheat dominates\cite{ISI:000376712600013,Landberg2009}.



%%%%%%%%%%% WHEAT INTERVENTION


%In \acrfull{wg} source dominated by wheat, total ARs got good correlation with wheat intake. e.g. UK and America. 

%Meanwhile, ARs metabolites could also be a potential marker, but may not be a good marker in mixed-ARs source countries since ARs metabolites were not specific to WG wheat. 
%ARs from WG rye could also be metabolized to same products causing confounding.
%%%%%%%%%% WHEAT OBS
%\begin{table}[h!]
%	\scalebox{0.83}{
%		\begin{tabular}{|c|c|c|c|c|c|c|}
%			%header
%			\hline 
%			\makecell{Type of\\ WG} & \makecell{No.\\subjects} & \makecell{Sample\\type}  & \makecell{Analytical\\method} & \makecell{Candidate\\biomarker(s)} & \makecell{Associated\\with} & Reference \\ 
%			\hline
%			%%EU cohort
%			WGs\footnote{This cohort studies investigated WGs conc. in different EU countries' population.} & 2845 & \makecell{\makecell{fasting and \\non-fasting\\plasma}}  & GC-MS & AR C17:0/C21:0 & FFQ & \cite{ISI:000334172400017} \\ 
%			\hline	
%			
%	\end{tabular}}
%	\caption{Biomarkers of Wheat Intake Reported in Observation study}
%	\label{table:wheat_observation}
%\end{table}
%In this study\cite{ISI:000298402100026}, total ARs, rather than the ratio (C17:0/C21:0), were reported as \acrshort{wg} wheat intake biomarkers.
%This intervention was conducted in UK. In British population, the major whole grain source is wheat. Rye was rarely consumed. Therefore, plasma ARs got a good correlation with \acrfull{wg} wheat intake.

\subsubsection{Applications in Type II Diabetes Research}
Combining plasma total ARs concentration and the ratio (C17:0/C21:0) can subjectively estimate \acrshort{wg} wheat and rye intake and approximate their relative compositions in the diet. This will provide information of specific cereal species intake amount and facilitate their disease preventative effects study.

For example, an observation study showed, in Chinese population AR metabolite DHPPA was correlated with lower odds of type II diabetes and impaired glucose regulation\cite{ISI:000430455900021}. However, DHPPA as AR metabolite can come from either whole grain wheat or rye, therefore not specific to WG wheat.

Other two researches can better show that, actually \acrshort{wg} rye could be more favorable for type II diabetes.  An observational study showed in a population with metabolic syndrome, plasma AR C17:0/C21:0 was associated with increased insulin sensitivity\cite{ISI:000333777700008}. Further, it was observed that in healthy Scandinavian populations, plasma total ARs concentration was not correlated with type II diabetes risk. However higher C17:0/C21:0 ratio (implicating more rye intake) was associated with increased insulin sensitivity\cite{ISI:000378977200013}. These results imply that, a whole grain diet dominated by rye could be favourable for type II diabetes prevention. 

In EPIC cohort study\cite{ISI:000334172400017}, there is an interesting phenomena. Rye has higher constitutions in Danish populations's WG source (70\%) than Swedish (55\%) on average. Regarding C17:0/C21:0 ratio, Danish participants in EPIC cohort showed lower value (0.37) than Swedish participants (0.43). 
However, Swedish participants were healthy adults, while participants from Denmark in EPIC study were obese or over-weights subjects. Those participants may have different dietary habits and consume less rye than average Danes. This may imply rye could also be favorable in weight control.

These studies indicate that, application of biomarkers for sub-type cereal intake could provide more information to study each cereal subtype's health beneficial effects.

\subsubsection{Other potential markers}
Searching results also showed some \textit{Food compound intake biomarkers (FCIBs)} research as defined by Gao\cite{Gao2017} such as phenolic compounds\cite{ISI:000389134200003}, benzoxazinoids (BXOs)\cite{ISI:000394168100034,ISI:000348343300015}, phytoestrogen\cite{ISI:000384082300001}, phytosterol and lignan\cite{ISI:000387249200001} and\textit{effect markers} Dragsted\cite{Dragsted2017} microbioal derivitives\cite{ISI:000348343300015}. These compounds also present in other food,  not specific for \acrshort{wg} wheat. These results were summarized in Appendix.

In this article\cite{ISI:000387249200001}, author performed an intervention study using refined wheat bread as control group, authors proposed a panel of metabolites consisting 7 AR metabolites, 5 BXO metabolites and 5 phenolic acid derivatives can be used to objectively monitor WG wheat intake. The concentration of these phytochemicals varied in different cereal grains. Therefore, a combination of their metabolites could potentially indicate intake of different cereals. However, this conclusion needs to be further identified in other intervention and observation studies.


