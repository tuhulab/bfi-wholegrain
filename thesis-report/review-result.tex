\subsection{WG Barley}
The literature search did not find any direct results about WG barley intake biomarker research. This is probably due to limited dietary exposure of barley in population. Although barley is the 4th most produced cereal grains worldwidely. Most of them is used for brewing or feed. Only approximately 4\% is consumed directly\cite{Baik2008}.

One intervention study\cite{DeAngelis2015} showed in fecal samples, barley intake can significantly change 

Several observation studies investigated correlation between ARs metabolites and whole grain intake. Although they tried to cover more whole grain species, for example, one study 

A lot of research were interested in barley's effect biomarkers. 

 However, barley was not solely listed. Therefore, although ARs and their metabolites got good correlation results with these 'Whole-grain intake'. Whether ARs and their metabolites can indicate WG barley intake needs to be validated.

based on another search results, showed that ARs concentration in WG barley flour has similar concentration 

in order to figure out more subjective barley intake, it seems inappropriate to use ARs and their metabolites to quantify.\footnote{My personal viewpoint, how do you consider, Lars?}.


However, from beer intake biomarker article\cite{Gurdeniz2016}, Hordenine could be a potential biomarker.


\subsection{WG Wheat}