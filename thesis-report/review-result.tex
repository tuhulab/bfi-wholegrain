\subsection{WG barley}
The literature search got 129 records after removing duplicate records from merged 3 database search results. Within them, none of the studies directly investigated WG barley intake biomarkers. This could be explained by limited dietary exposure of barley in population. Although barley is the 4th most produced cereal grains worldwidely. Most of them is used for brewing or feed. Only approximately 4\% is consumed directly\cite{Baik2008}.

A 2-month intervention study\cite{DeAngelis2015} incorporated 75\% refined drum wheat and 25\% WG barley. The fecal samples showed significant change in microbiota and metabolome after intervention\cite{DeAngelis2015}. However, no specific metabolite can indicate WG barley intake.

ARs and their metabolites may not indicate WG barley intake. Several observation studies\cite{ISI:000309032000011,ISI:000259554500019} investigated correlation between ARs metabolites and whole grain intake. Although these studies tried to cover many whole grain species, for example, one study\cite{ISI:000259554500019} listed 7 types of regularly consumed WGs in American populations in the FFQ (Food Frequency Questionnaire)\footnote{Dark breads, High-fiber or bran cereals, Cooked cereals and grits, Regular granola, Granola bars and cereal bars, Plain popcorn (no butter) or low-fat microwave popcorn, Buttered or gular microwave popcorn}, barley was not solely listed. Therefore, although ARs and their metabolites got good correlation with these 'Whole-grain intake'. Readers should be cautious to apply them to WG barley. In addition, ARs concentration in cereal barley is much lower compared with WG wheat and rye, with similar concentration with refined wheat and rye flours (Table-\ref{table:ars_in_plant}). 
\begin{table}[h!]
		\footnotesize
\centering

\begin{tabular}{|c|c|c|c|c|c|}
	\hline 
	Cereal & \makecell{Conc. range \\in cereal}& \makecell{Conc. average \\or range \\in WG flour }& \makecell{Conc. average \\in refined \\flour} & \makecell{Main\\homologues} & \makecell{C17:C21\\ homologues\\ratio} \\ 
	\hline 
	Rye&360-3200& 972 &90&C17, C19, C21& 0.8-0.9 \\ 
	\hline 
	Wheat&761-8390& 490-710 & 36& C19, C21& 0.07-0.1 \\ 
	\hline 
	Barley& 55.8-98.2&NA & NA& C19, C21, C25&NA  \\ 
	\hline 
\end{tabular} 
	\caption{Prensence of ARs in Cereal Grains, adapted from \cite{doi:10.1021/jf0340456,ANDERSSON2010794,BORDIGA201638}(unit: \SI{}{\micro\gram}/g dm)}
	\label{table:ars_in_plant}
\end{table}


Most research results focused on barley's \textit{effect biomarkers} as defined by Dragsted\cite{Dragsted2017} and Gao\cite{Gao2017}, such as bowel health indicators, lipid profiles and cardiovascular disease (CVD) markers, etc.

%From food chemistry, plant and cereal science articles, beer intake biomarker article\cite{Gurdeniz2016}, Hordenine could be a potential biomarker.
Further search results in food chemistry, cereal science and plant science involved some compounds only present in barley other than other food. These could give hints for further identification. The results were summarized in Table-\ref{table:candidate_biomarker_barley}.


\begin{table}
	\centering
	\small
\scalebox{0.78}{
\begin{tabular}{|c|c|c|c|c|c|}
	\hline 
	No & \makecell{Candidate\\biomarker}& Formula & \makecell{Chemical\\group} &Presence in Food & Reference\\ 
	\hline 
	1&Hordenine& C10H15NO &alkaloid&\makecell{germinating barley,\\ beer and other plants}&\cite{Gurdeniz2016}\\ 
	\hline
	4&Hordatine A&C28H38N8O5&alkaloid&\makecell{only reported\\in barley}&FoodDB(002330)\\ 
	\hline 
	4&Hordatine B&C29H40N8O5&alkaloid&\makecell{only reported\\in barley}&FoodDB(002328)\\ 
	\hline 
	2&\makecell{Distichonic\\acid A}&C10H18N2O8&\makecell{gamma amino acids\\ and derivatives}&\makecell{only reported\\in barley}&FoodDB(18164)\\
	\hline 
	3&\makecell{Distichonic\\acid B}&C10H18N2O8&\makecell{gamma amino acids\\ and derivatives}&\makecell{only reported\\in barley}&FoodDB(018165)\\ 
	\hline 
	
	5&\makecell{14,16-Nona\\cosanedione}&C29H56O2&ketone&\makecell{only reported\\in barley}&FoodDB(013891)\\ 
	\hline 
	6&N-Norgramine&C10H12N2&indole&\makecell{only reported\\in barley}&FoodDB(017815)\\ 
	\hline 
\end{tabular} }
\caption{Candidate Biomarkers for WG barley intake}
\label{table:candidate_biomarker_barley}
\end{table}

To conclude, barley, especially WG barley attracted a lot of interest due to its health beneficial effects for chronic disease. However, due to barley's limited exposure, currently there's no biomarkers can indicate its intake. However, a lot of sparse information was reported from cereal and food chemistry could further benefit identification and validations of WG barley's intake biomarkers.

\subsection{WG wheat}
The literature search got 312 results after removing duplicate records from merged results. XXX were used.

ARs and their metabolites were widely reported, validated and applied biomarkers for WG wheat and rye intake. Depending on different processing methods and grain species, ARs concentration varied in WG rye and wheat. In order to distinguish wheat and rye intake, the ratio C17:0/C21:0 was proposed as an biomarker to indicate in dietary pattern which cereal dominates. if this ratio is close to 1.0, then rye dominated. If the ratio is close to 0.1, then wheat dominated\cite{ISI:000376712600013}. 

\begin{table}[h!]
\begin{tabular}{|c|c|c|c|c|c|c|}
	\hline 
	\makecell{Type of\\ WG} & \makecell{No.\\subjects}  & \makecell{Sample\\type}  & \makecell{Analytical\\method}& \makecell{Candidate\\biomarker(s)} & Reference \\ 
	\hline 
	\makecell{WG wheat\\WG rye} & 39 & plasma & GC-MS & AR C17:0/C21:0 & \cite{ISI:000376712600013} \\ 
	\hline 
\end{tabular} 
\caption{Potential Biomarkers of Wheat Intake in Intervention study}
\label{table:wheat_intervention}
\end{table}


\textit{Food compound intake biomarker} as defined by Q. Gao etc\cite{Gao2017}.
phenolic compounds\cite{ISI:000389134200003}, benzoxazinoids\cite{ISI:000394168100034} and phytoestrogen\cite{ISI:000384082300001}.
