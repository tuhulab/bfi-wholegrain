\documentclass[]{article}

%opening
\title{Course Syllabus \\ {\small Crash Course on LC-MS Based Metabolomics Data Preprocessing}}

\author{Tu Hu\footnote{Data analyst, tu@nexs.ku.dk}}
\date{Apr, 2019}
\usepackage{graphicx}
\usepackage{makecell}
\begin{document}

\maketitle

\section{Contents}
	In this crash course, participant(s) will be briefly introduced the principles of LC-MS based metabolomics, principles of data-preprocessing and general procedures. 
	Further, data from participant(s)' project will be used as an example and worked out by MZmine 2 with guidance. Final results will be a feature table including file name (or sample name), retention time, m/z and intensities for further analysis.
\section{Learning Outcomes}

After completing the course, the student(s) should have:
\\

\textbf{Knowledge about:}
\begin{itemize}
	\item Principles of LC-MS based untargeted metabolomics
	\item Principles of data preprocessing of LC-MS based untargeted metabolomics\\
\end{itemize}


\textbf{Skills in/to:}
\begin{itemize}
	\item Explore metabolomics data by MassLynx
	\item Applying MZmine 2 to preprocess LC-MS based untargeted metabolomics data 

\end{itemize}


\textbf{Competences in/to:}
\begin{itemize}
	\item Independently handle LC-MS based metabolomics data preprocessing task
\end{itemize}

\newpage
\section{Course Structure}
\begin{table}[h]
\scalebox{0.83}{

\begin{tabular}{|c|c|c|c|c|}
	\hline 
	& Contents & Time & Location & Duration \\ 
	\hline
	\makecell{Hands-on\\exercise 0} & \makecell{Explore Data on MassLynx} & \makecell{9 am \\29th, Apr} & Metabolomics Lab & 20 min \\ 
	\hline 
	Lecture & \makecell{Principles of \\LC-MS based Metabolomics\\and data preprocessing} & \makecell{upon\\participant's\\reservation} & Local meeting room & 20 min \\ 
	\hline 
	\makecell{Hands-on\\exercise 1} & Real case work & \makecell{upon\\participant's\\reservation} & Local meeting room & 1 h \\ 
	\hline 
	\makecell{Hands-on\\exercise 2}  & Real case work & \makecell{upon\\participant's\\reservation} & Local meeting room & 1 h \\ 
	\hline 
	\makecell{...}  & \makecell{Dependent on participant's progress}& \makecell{upon\\participant's\\reservation} & Local meeting room & ... \\ 
	\hline 
	Q \& A & \makecell{Questions regarding \\data preprocessing} & \makecell{upon\\participant's\\reservation} & ad-hoc & 1 h \\ 
	\hline 
\end{tabular} }
\end{table}
\section{Reading Materials and Guides}
Reading materials will be delivered to student's email address in pdf. Other formats (such as Mendeley entries, LaTex citation commend or BibTex entry) are available upon request.

1. Yi, L. et al. Chemometric methods in data processing of mass spectrometry-based metabolomics: A review. Anal. Chim. Acta 914, 17–34 (2016).\\\textbf{\textit{Only 2.1 (Pre-processing of raw data) is mandatory.}}\\


2. Pluskal, T., Castillo, S., Villar-Briones, A. \& Orešič, M. MZmine 2: Modular framework for processing, visualizing, and analyzing mass spectrometry-based molecular profile data. BMC Bioinformatics (2010). doi:10.1186/1471-2105-11-395 \\\textbf{\textit{Only Figure 1 is mandatory.}}\\

3. MZmine 2 Downloading Page and Installation Manual: http://mzmine.\\github.io/download.html \\\textbf{\textit{It is recommended to download and familiarize this software before starting the crash course. If installation problems are encountered, they will be troubleshot in hands-on exercise.}}\\

4. Karaman, I., Climaco Pinto, R. \& Graça, G. Metabolomics Data Preprocessing: From Raw Data to Features for Statistical Analysis. Compr. Anal. Chem. 82, 197–225 (2018). \\\textbf{\textit{Following sections are optional: \\3. NMR Preprocessing (However, participants should aware that they may be asked about NMR in thesis defence by opponents) \\4.1.2.4 LC-MS Preprocessing Example}}\\

5. MZmine Instructions by Gözde Gürdeniz\\\textbf{\textit{The most important reading material for this crash course. Participant(s) will be guided through this instruction.}}\\

6. LC-MS Data Preprocessing lecture slides by Gözde Gürdeniz from PhD course Introduction to Nutitional Metabolomics\\\textbf{\textit{This is a very detailed overview of LC-MS Metabolomics data preprocessing.}}\\
\section{Participant(s)}
Marina Roberts (lhw351@alumni.ku.dk; m.roberts180@gmail.com\footnote{Prefer to be contacted by gmail.})

\section{Remarks}
Participant(s) are expected to understand the principles of LC-MS based metabolomics. If not, they will be briefly introduced with a vivid analogy.

This course was developed under the supervision of Lars Ove Dragsted and Natalia Manjarrez. 

\end{document}
