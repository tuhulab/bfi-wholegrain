\subsection{Geographical Applicability and Specificity of Whole Grain Wheat Intake Biomarkers}
Biomarkers for \acrshort{wg} wheat intake could have their geographical applicabilities.
Here, "geography" refers to an area having similarly available food resources.

In the case of WG wheat intake, some biomarkers were reported to be capable of distinguishing whole grain wheat intake in US population and proposed as putative intake biomarkers, including DHBA, DHPPA, DHBA glycine, 3,5-DHPPTA\cite{ISI:000330080500002}, ARs (C19:0, C21:0, C23:0)\cite{ISI:000374112900032}. However, according to current criterial of BFIRev\cite{Pratico2018} and validation\cite{LarsValidation}, these biomarkers were not included as "candidate biomarker" of whole grain wheat intake because of their non-specificities:
\begin{itemize}
	\item DHBA, DHPPA, DHBA glycine, 3,5-DHPPTA: These are AR metabolites. When \acrlong{ars} are metabolized, information of their homologue ratio can not be traced. Therefore, it's impossible to distinguish ARs are from WG wheat or WG rye.
	\item ARs (C19:0, C21:0, C23:0): WG wheat mainly consists these three homologues. They had good correlation in an intervention study\cite{ISI:000374112900032}. However, these compounds could also come from WG rye.
\end{itemize}

To our best knowledges, rye is rarely consumed in US. Therefore, in such a "geography" where WG wheat is the only AR source. These markers have their geographical applicabilities although are not be world-widely applicable. 
\begin{table}[h!]
	\scalebox{0.55}{
		\begin{tabular}{|c|c|c|c|c|c|c|c|c|}
			\hline
			
			
			
			\makecell{Food\\items} &\makecell{No.\\subjects}& \makecell{Study\\design} &  \makecell{Sample\\type}  & \makecell{Analytical\\method}& \makecell{Candidate\\biomarker(s)} & Identifier & Reference \\ 
			\hline 
			
			\makecell{WG wheat bread\\Refined wheat bread}& 12& \makecell{Intervention} & \makecell{24-h\\urine}  & HPLC-CE\footnote{coulometri electrode array detector} & \makecell{DHBA\\DHPPA\\DHBA glycine\\3,5-DHPPTA} & \makecell{HMDB0013677\\HMDB0125533\\InChi: QVGDKHUNWDVPOR-\\UHFFFAOYSA-N\\InChi: QHXNJIMVPAFCPR-\\UHFFFAOYSA-N} & \cite{ISI:000330080500002} \\ 
			\hline	
			
			\makecell{WG wheat\\(3 or 6 servings)} & 19 & \makecell{Intervention\\(crossover\\1 week)} & fasting plasma & GC-MS & \makecell{ARs\\(C19:0,C21:0,C23:0)} & \makecell{HMDB0030956\\HMDB0031035\\HMDB0038524} &\cite{ISI:000374112900032} \\ 
			\hline 
			
			
	\end{tabular} }
	\caption{"Geographically applicable (specific)" Biomarkers of Whole Grain Wheat Intake}
	\label{table:wheat_intervention}
\end{table}

\subsection{Necessities of Discovering Biomarkers for WG Wheat Intake}
In order to clarify each sub-type of cereal's health beneficial effects, it is important to accurately quantify exposure amount of each sub-type cereal. BFIs showed their strengths and potentials in studying WGs. 

it is essential to discover intake biomarker for each sub-type cereal grain.
Currently, most studies showed interest in WG effect biomarkers.

As discussed in \cite{ISI:000447355100002}, one of the challenges in BFIs discovery of WG is that the chemical compositions of most of WGs were not systematically studied. due to limited systematic research on phytochemicals 

%%%%%%
more studies should be conducted to investigate WG sub-type's intake biomarkers. because each whole grain sub-type may not benefit health equally in different population.

It worths pointing out that, in Chinese population WG study, to the author's knowledge, WG rye is rarely produced and consumed in Chinese population, therefore, DHPPA could primarily come from WG wheat \footnote{A letter has been sent to the author suggesting them to investigate the AR homologue ratio in plasma samples. }
interesting conclusions may be drawn. WGs's health beneficial effects may be different in different populations.

In scandinavian population, WG rye and wheat. . therefore, in order to investigate intake biomarker for each sub-type cereals, single WG source country or WG low consumption country such as China could be a good choice.

%%database (MAYBE NOT THE SCOPE OF THIS REVIEW ARTICLE??)
not too many food compound and natural product, phytochemical database are available.

Using ARs to represent WG intake.
%%another question we observed in some observation studies is, some researches use ARs indicate total WG intake and got good correlation. ARs can not indicate WG wheat and rye intake. 
%%Althought it sounds possible, people consumed more whole grain rye and wheat may also consume other whole grain cereals and achieve better health benefical effects. but this correlation does not nessarily exist.
%%People eat more WG wheat and rye do not necessarily mean they eat a lot corns for example. therefore, errors, deviations or confounding can be generated due to lack of more specific markers for each whole grain species.
%%for example, in USA, WG corn is also an important 