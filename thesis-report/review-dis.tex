\subsection{Potential BFIs for WG barley}
Search results in food chemistry, cereal science and plant science database showed some compounds presenting exclusively in WG barley. These could give hints for further identification. The results were summarized in Table-\textbf{\ref{table:candidate_biomarker_barley}} in \textbf{Appendix}.


\subsection{Exclusion Criteria of WG Wheat intake biomarker}
Most intervention studies used \acrshort{wg} diet containing several types of cereals as a comparison with refined diet within searched results. The reason could be, most people regularlly consume more than one WG type. Therefore, research interests were also in the combination of several WG types. 

We used a strict exclusion criteria in order to identify specific biomarkers to distinguish WG wheat intake from refined wheat or other WG types:
\begin{enumerate}
	\item The terms ``WGs or dietary fibers or cereal fibers or bread" were used. However the details about how much WG wheat wasn't included.
	\item In observational studies, food intake was only measured by biomarkers without any validations (such as FFQ, FR etc)
	\item Interventions were WG diet or meal (including several WG types). Biomarkers were not specific to wheat.
\end{enumerate}

%% NOT QUITE SURE IT'S CORRECT OR NOT. 

%most commonly using \acrshort{wg} rye and wheat and reporting ARs as intake biomarkers for \acrshort{wg} rye and wheat. 
%Very few intervention studies investigated biomarkers for different SUBTYPE whole grains' intake. 

%% My research question is to find marker specific for WG wheat.
%% OBSERVATIONAL STUDY
Another excluded example is ambiguity of cereal types\cite{ISI:000348343300015}, subjects were classified as consumers of "none-bread", "white bread" or "whole grain bread". However, no further details about cereal types of the WG bread.

In observational studies, estimations of food exposure are majorly self-report based. 
In these self-based surveys, participants had difficulty recalling and distinguishing the different cereal species. Results were normally not detailed to each WG type.
Therefore, it is difficult to assign the biomarker to specific WG. 
Food frequency questionnaire causes high deviations distingushing each sub-type cereal by its nature.

Those non-specific markers were listed in appendix. 

\subsection{AR Metabolites for WG Wheat Intake and Geographical Applicabilities}
Biomarkers for \acrshort{wg} wheat intake could have their geographical applicabilities. In this context, ``geography" refers to a region having similarly available food resources.

In the case of WG wheat intake, some biomarkers were reported to be capable of distinguishing whole grain wheat intake in US population and proposed as putative intake biomarkers, including DHBA, DHPPA, DHBA glycine, DHPPTA\cite{ISI:000330080500002}, ARs (C19:0, C21:0, C23:0)\cite{ISI:000374112900032}. However, according to current criterial of BFIRev\cite{Pratico2018} and validation\cite{LarsValidation}, these biomarkers were not classified as "candidate biomarker" because of their non-specificities:
\begin{itemize}
	\item DHBA, DHPPA, DHBA glycine, DHPPTA: They are phase I metabolites of AR(Fig-\ref{fig:structure_ars}) and detected in urine. When \acrshort{ars} are metabolized, information of their homologue ratio can not be traced. Therefore, it's impossible to distinguish their origins are WG wheat or WG rye.
	\item ARs (C19:0, C21:0, C23:0): WG wheat mainly consists these three homologues. They were reported well correlated with WG wheat intake in an intervention study\cite{ISI:000374112900032}. However, WG rye intake can also produce these metabolites.
\end{itemize}

To our best knowledges, rye is rarely consumed in USdue to its limited availabilities. Therefore, in such a ``geography" where WG wheat is the only AR source. These markers have their geographical applicabilities though they might not be applied in both WG rye and wheat consuming areas. 

%Validation in another population
%This also raised another question about biomarker validation. Regarding confidence level of biomarkers, when a "putative" biomarker can be upgraded to "candidate" biomarker, it needs to be confirmed by more trails and preferably by another design or in a different population. Therefore, this may not fit for these "geographically-applicable" biomarkers. 

\begin{figure}[h!]
	\centering
	\includegraphics[width=0.5\linewidth]{picture/ars_sang_pathway}
	\caption{Structure of \acrshort{ars} and suggested metabolic pathway, adapted from \cite{ISI:000447355100002}}
	\label{fig:structure_ars}
\end{figure}

\begin{table}[h!]
	\scalebox{0.55}{
		\begin{tabular}{|c|c|c|c|c|c|c|c|c|}
			\hline			
			\makecell{Food\\items} &\makecell{No.\\subjects}& \makecell{Study\\design} &  \makecell{Sample\\type}  & \makecell{Analytical\\method}& \makecell{Candidate\\biomarker(s)} & Identifier & Reference \\ 
			\hline 
			
			\makecell{WG wheat bread\\Refined wheat bread}& 12& \makecell{Intervention} & \makecell{24-h\\urine}  & HPLC-CE\footnote{coulometri electrode array detector} & \makecell{DHBA\\DHPPA\\DHBA glycine\\3,5-DHPPTA} & \makecell{HMDB0013677\\HMDB0125533\\InChi: QVGDKHUNWDVPOR-\\UHFFFAOYSA-N\\InChi: QHXNJIMVPAFCPR-\\UHFFFAOYSA-N} & \cite{ISI:000330080500002} \\ 
			\hline	
			
			\makecell{WG wheat\\(3 or 6 servings)} & 19 & \makecell{Intervention\\(crossover\\1 week)} & fasting plasma & GC-MS & \makecell{ARs\\(C19:0,C21:0,C23:0)} & \makecell{HMDB0030956\\HMDB0031035\\HMDB0038524} &\cite{ISI:000374112900032} \\ 
			\hline 
			
	\end{tabular} }
	\caption{``Geographically applicable (specific)" Biomarkers of Whole Grain Wheat Intake}
	\label{table:wheat_intervention}
\end{table}

%\subsection{Analytical performances of ARs}
%In most of studies, \acrshort{ars} were analysed by GC-MS.
%Analytical process included chemical derivatization prior to injections and a high collision energy.

%Whether phase II metabolites may interfere quantification results is unknown.
%This could be investigated more to improve this markers' analytical stabilities.

%\subsection{Necessities and Challenges of Discovering Biomarkers for WG Wheat Intake}
%In order to clarify health beneficial effect of each cereal subtype, it is important to accurately quantify exposure amount of each sub-type cereal. BFIs showed their strengths in studying WGs. Discovering intake biomarker for each sub-type cereal grain becomes necessary. Currently, most studies showed interest in WG effect biomarkers.

%One of the challenges in BFIs discovery of WG is chemical compositions of most of WGs were not systematically studied. 
%Food compound, natural product, phytochemical database also need to be expanded.

\subsection{Ambiguous Use of the Term ``Biomarker"}
Ambiguously using the term ``biomarker" in the publications hurdled scientific communications. 
Most retrieved results mentioned ``biomarker". However, ``biomarkers" referred different concepts in different contexts, e.g. \textit{food intake biomarkers}, \textit{food compound intake biomarkers} and \textit{effect biomarkers}. In literature search phase of this review, it is difficult for us to quickly classify biomarkers before reading the full text. Correctly using these terms might reduce confusions and make communications easier. 

Gao\cite{Gao2017} and Dragsted\cite{Dragsted2017} proposed the new ontology and classification schema of ``Biomarker". The awareness and implementations of the new ontology and classification schema will relief this problem.



%%%%%%
%more studies should be conducted to investigate WG sub-type's intake biomarkers. because each whole grain sub-type may not benefit health equally in different population.

%It worths pointing out that, in Chinese population WG study, to the author's knowledge, WG rye is rarely produced and consumed in Chinese population, therefore, DHPPA could primarily come from WG wheat \footnote{A letter has been sent to the author suggesting them to investigate the AR homologue ratio in plasma samples. }
%interesting conclusions may be drawn. WGs's health beneficial effects may be different in different populations.

%In scandinavian population, WG rye and wheat. . therefore, in order to investigate intake biomarker for each sub-type cereals, single WG source country or WG low consumption country such as China could be a good choice.

%%database (MAYBE NOT THE SCOPE OF THIS REVIEW ARTICLE??)
%not too many 

%Using ARs to represent WG intake.
%%another question we observed in some observation studies is, some researches use ARs indicate total WG intake and got good correlation. ARs can not indicate WG wheat and rye intake. 
%%Althought it sounds possible, people consumed more whole grain rye and wheat may also consume other whole grain cereals and achieve better health benefical effects. but this correlation does not nessarily exist.
%%People eat more WG wheat and rye do not necessarily mean they eat a lot corns for example. therefore, errors, deviations or confounding can be generated due to lack of more specific markers for each whole grain species.
%%for example, in USA, WG corn is also an important 