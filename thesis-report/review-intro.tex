Whole grains (WGs) and their processed food contain a lot of non-nutrient compounds in their bran parts. Besides providing carbonhydrates which mostly locate in grains' endosperm, WGs may have other health beneficial effects such as disease prevention. However, epidemiologic studies showed mixed results due to subjective self-report based food exposure measurement\cite{ISI:000447355100002}. Using \acrfull{bfis} can potentially measure food exposure in population more objectively with accuracies and details\cite{Scalbert2014}.

Alkylresorcinols (ARs) and their metabolites were widely reported and validated biomarkers for \acrshort{wgs} intake. 
In plants commonly consumed for food, \acrshort{ars} only present high amounts in rye and wheat, especially concentrated in their bran parts\cite{arreview2004}. Therefore, \acrshort{ars} have the possibility to be used as biomarkers for whole grain wheat and rye intake.

Increasing evidence showed that, different \acrshort{wg} cereal types (such as wheat, rye, oat, barley etc.) could benefit health differently. 
However, classical self-reported measurement tools used in observational studies could cause biases and confoundings to distinguish each cereal type.
Therefore, discovering \acrshort{bfis} of each whole grain type could potentially provide a tool to accurately quantify their exposures. 

This mini-review aimed at systematically examining available literatures to obtain information of potential biomarkers for WG barley and wheat intake. This will prioritize further identification and validation of the thesis work.