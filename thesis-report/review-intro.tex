Whole grains (WGs) and their processed food could have health beneficial effects. However, epidemiologic studies showed mixed results due to subjective self-report based food exposure measurement\cite{ISI:000447355100002}. Using biomarkers of food intake (BFIs) can potentially measure food exposure in population more objectively with accuracy and detail\cite{Scalbert2014}.

Alkylresorcinols (ARs) and their metabolites were widely reported and validated biomarkers for WGs intake. 
In plants commonly consumed for food, ARs only present in high amounts in rye and wheat, especially concentrated in their bran parts\cite{arreview2004}. Therefore, ARs have the possibility to be used as biomarkers for whole grain wheat and rye intake.

Increasing evidence showed that, different cereal types could benefit health differently. Therefore, discovering biomarkers of each whole grain sub-type exposure could be helpful to better understand each cereal type's health beneficial effects. Hence, better dietary guidelines could be suggested to the public.

This mini-review aimed at systematically examining available literatures to obtain information of potential biomarkers for WG barley and wheat. This will prioritize further identification and validation of the thesis work.
