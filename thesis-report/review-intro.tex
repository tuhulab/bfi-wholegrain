Whole grains (WGs) contain a lot of non-nutrients in the bran. 
These non-nutrients might benefit health. Several dietary guidelines suggested WG intake\cite{Piepoli2016}. A recent meta-analysis confirmed high intake of WGs is associated with reduced risk of cardiovascular disease, cancer, and all cause and cause specific mortality\cite{Aune2016}. 

However, recommendations for \acrshort{wg} intake have often been unclear or inconsistent with regard to the amount and types of whole grain foods that should be consumed to reduce chronic disease and risk of mortality\cite{Aune2016}. Meanwhile, increasing evidence showed different \acrshort{wg} types (such as wheat, rye, oat, barley etc.) could benefit health differently. 

Classical self-reported measurement tools (e.g. food diaries and food frequency questionnaires) used in observational studies could cause biases and confoundings in differentiating each cereal type.
 
 %due to subjective food exposure measurement based on self-report\cite{ISI:000447355100002}. Using \acrfull{bfis} can potentially measure food exposure in population more objectively with accuracies and details\cite{Scalbert2014}.
%Alkylresorcinols (ARs) and their metabolites were widely reported and validated biomarkers for \acrshort{wgs} intake. 
%In plants commonly consumed for food, \acrshort{ars} only present high amounts in rye and wheat, especially concentrated in their bran parts\cite{arreview2004}. Therefore, \acrshort{ars} have the possibility to be used as biomarkers for whole grain wheat and rye intake.

Discovering \acrshort{bfis} of each whole grain type (e.g wheat, barley, rye, corn, rice etc.) could potentially provide a tool to accurately quantify their exposures. WGs' health beneficial effects could be further elucidated. 

This mini-review aimed at systematically examining available literatures to obtain information of potential biomarkers for WG barley and wheat intake. This will prioritize further identification of the thesis work.