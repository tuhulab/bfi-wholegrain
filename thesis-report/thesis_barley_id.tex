\section{Abstract}

\section{Introduction}
\subsection{Phytosterol: Structure, Post-harvest Degradation and Fragmentation Behaviour}
Phytosterols ubiquitously occur in plant-based food\cite{sterolmsms}. They were claimed to have health beneficial effects, such as lowering cholesterol. They occur in food as free sterols (FS), steryl esters (SE), and glycosylated conjugates comprised of steryl glucosides (SG) and acylated steryl glucosides (ASG).

Specific sterol profiles characteristic to certain plant families have been identified showing that a broad range of minor sterols occurs as free sterols or glycosylated conjugates

This ion could be stanol (a sub-type of phytosterol with a saturated B-ring) derivative inferred from its C-ring fragmentation behaviour: 
\begin{itemize}
	\item higher intensities of m/z 149 than both 147 and 145
	\item higher intensities of m/z 161 than both 159 and 163
\end{itemize}
 

\subsection{$\beta$-glucuronidase for metabolites identification}
%Glucuronide group hurdles the structure elucidation of metabolites.
%therefore, in structure elucadation stage, in order to confirm the structre. it's recommended to hydrolyze the glucuroide group.
%Although there are a lot of studies trying to elucidate sturucture of phase II metabolites (i.e. glucuronate), the exact conjugation site can only be determined by additional NMR analysis. 

%A lot of metabolites are glucuronates? 
%Which reaction is it catalysing?
$\beta$-glucuronidase catalyses $\beta$-glucuronates hydrolysis. This enzyme is routinely used for enzymatic hydrolysis of urine, plasma and other fluids prior to analysis by enzyme immunoassay or mass spectrometry etc.

$\beta$-glucuronates are common phase II metabolites presenting in urine and plasma.
\section{Materials and methods}
\subsection{Chemicals}
Sitostanol reference compound (CAS: 83-45-4, Avanti Polar Lipids Inc., USA), $\beta$-glucuronidase (CAS: 9001-45-0, E.C. number 3.2.1.31, Sigma-Aldrich, from \textit{Escherichia Coil})

\subsection{Apparatus}
%UPLC-MS system (column C18, QTOF (VION, Waters, Milford, USA), Water Bath

\subsection{$\beta$-glucuronidase experiment}
Urine samples were treated with $\beta$-glucuronidase for 1.5 h to hydrolyse glucuronate group. A positive control was used. Details were described in \textbf{Appendix}.

\subsection{UPLC-MS/MS analysis of Sitostanol}
%\cite{doi:10.1021/jf501509m}
%Mobile phase A: H2O
%Mobile phase B: Methanol
%Binary methods developed and optimized 
%Analysis methods referred\cite{sterolmsms}. Sitostanol stock solution (1 mg/mL) was prepared in 100\% ethanol. Further, stock solution was diluted by methanol, with the concentration of 0.02 mg/mL. 
%ESI positive mode was used to ionize.
\section{Results}
\subsection{Retention Time (RT) and m/z in Different Matrix of Sitostanol}
\begin{tabular}{|c|c|c|c|}
	\hline 
	Matrix & RT & m/z (ESI+) & Annotation \\ 
	\hline 
	Whole Grain Barley & 6.88 & 291.2683 & Unknown \\ 
	\hline 
	Urine & 6.71 & 291.2683 & Unknown \\ 
	\hline 
	Standard & 8.60 & 399.3989 & [Sitostanol-H2O+H]- \\ 
	\hline 
\end{tabular} 

\subsection{$\beta$-glucuronidase Experiment}
$\beta$-glucuronidase hydrolysed both ions. The glucuronates decreased intensities. 

The expected unglucuronated ion (m/z 325.2739) was not detected. the reason could be: (1) not ionized.

Another expected unglucuronated ion (m/z 341.2675) was detected on the same day of experiment in full scan mode. 
However, this ion was not detected after one week storage in -20 freezer. This could be the reason of degradation. Ion 341.2675 was detected in RT 0.88 indicating it is a highly polar compound. Its structure needs to be further confirmed by MS/MS analysis.


\section{Discussion}


