\section{Abstract}

\section{Introduction}
\subsection{Phytosterol: Structure, Post-harvest Degradation and Fragmentation Behaviour}
Phytosterols ubiquitously occur in plant-based food\cite{sterolmsms}. They were claimed to have health beneficial effects, such as lowering cholesterol. They occur in food as free sterols (FS), steryl esters (SE), and glycosylated conjugates comprised of steryl glucosides (SG) and acylated steryl glucosides (ASG).

Specific sterol profiles characteristic to certain plant families have been identified showing that a broad range of minor sterols occurs as free sterols or glycosylated conjugates

This ion could be stanol (a sub-type of phytosterol with a saturated B-ring) derivative inferred from its C-ring fragmentation behaviour: 
\begin{itemize}
	\item higher intensities of m/z 149 than both 147 and 145
	\item higher intensities of m/z 161 than both 159 and 163
\end{itemize}
 

\subsection{$\beta$-glucuronidase for metabolites identification}
%Glucuronide group hurdles the structure elucidation of metabolites.
%therefore, in structure elucadation stage, in order to confirm the structre. it's recommended to hydrolyze the glucuroide group.
%Although there are a lot of studies trying to elucidate sturucture of phase II metabolites (i.e. glucuronate), the exact conjugation site can only be determined by additional NMR analysis. 

%A lot of metabolites are glucuronates? 

Which reaction is it catalysing?
$\beta$-glucuronidase are routinely used for enzymatic hydrolysis from urine, plasma and other fluids prior to analysis by enzyme immunoassay, mass spectrometry.
β-Glucuronidase (EC 3.2.1.31) catalyzes the reaction:
β-D-glucuronoside + H2O <--> D-glucuronate + an alcohol

\section{Materials and methods}
\subsection{Chemicals}
Sitostanol standard (CAS Number 83-45-4, Avanti Polar Lipids Inc., USA) was transported and stored in -20 \degree C. Ethanol

Beta-glucuronidase (CAS Number 9001-45-0, E.C. number 3.2.1.31, Sigma-Aldrich, from E. Coil, optimal pH 6-7)

\subsection{Apparatus}
UPLC-MS system (column C18, QTOF (VION, Waters, Milford, USA), Water Bath


\subsection{UPLC-MS/MS analysis of Sitostanol}
%\cite{doi:10.1021/jf501509m}
%Mobile phase A: H2O
%Mobile phase B: Methanol
%Binary methods developed and optimized 
%Analysis methods referred\cite{sterolmsms}. Sitostanol stock solution (1 mg/mL) was prepared in 100\% ethanol. Further, stock solution was diluted by methanol, with the concentration of 0.02 mg/mL. 
%ESI positive mode was used to ionize.

\subsection{Deconjugation experiments}
\begin{enumerate}
\item Prepare phosphate buffer\footnote{Calculated by AAT Bioquest} (0.1 M, pH=6.8, 50 mL): 
	\begin{itemize}
		\item Prepare 40 mL of distilled water in a volumetric bottle.
		\item Add 0.656 g of monosodium phosphate to the solution.
		\item Add 0.352 g of disodium phosphate to the solution.
		\item Adjust pH using HCl or NaOH.
		\item Add distilled water until volume is 0.05 L.
	\end{itemize}

\item Prepare enzyme solution (4 mg/mL): dissolve 0.0060 g $\beta$-Glucuronidase in 1.5 mL phosphate buffer. Store in -20\degree C freezer\footnote{Sigma Product Information: A solution in 75 mM phosphate buffer, pH 6.8, \textgreater 5 mg/ml may be stored at -20 \degree C for up to 2 months with little or no loss of activity.}
\item Prepare urine samples: thaw samples in the fridge and centrifuge (3000 rpm, 2 min). Prepare 2 Eppendor tubes, one labeled as 'blank', one as 'treatment'. Transfer 100 \si{\micro\litre} to each Eppendorf tube.
\item Enzymatic hydrolysis reaction: add 50 \si{\micro\liter} phosphate solution to 'blank', 50 \si{\micro\liter} enzyme solution to eppendorf tube, incubate in 37 \degree C for 1.5 h.
\item Denature enzymes to terminate the reaction
	\begin{itemize}
	\item Add 50 \si{\micro\liter} MeOH to the solution and vortex mix for 1 min
	\item Centrifuge at 3000 rmp for 3 min
	\item Add solvent A 300 \si{\micro\liter}
	\item Transfer supernatant to vials fur further LC-MS/MS analysis.
\end{itemize}


\end{enumerate}

\section{Results}
\subsection{Retention Time (RT) and m/z in Different Matrix}
\begin{tabular}{|c|c|c|c|}
	\hline 
	Matrix & RT & m/z (ESI+) & Annotation \\ 
	\hline 
	Whole Grain Barley & 6.88 & 291.2683 & Unknown \\ 
	\hline 
	Urine & 6.71 & 291.2683 & Unknown \\ 
	\hline 
	Standard & 8.60 & 399.3989 & [Sitostanol-H2O+H]- \\ 
	\hline 
\end{tabular} 

\subsection{Enzymatic Deglucuronidation}
Both ions were hydrolyzed by enzymes. 
In control group, ions have much higher intensities than treatment group.
However, their deglucuronited ions were not all detected.
ion 341.2675 was detected in RT 0.88 indicating it is a highly polar compound.

\subsection{}

