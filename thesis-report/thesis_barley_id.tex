\section{Abstract}

\section{Introduction}
\subsection{Phytosterol and Fragmentation Behaviour}
Phytosterols ubiquitously occur in plant-based food\cite{sterolmsms}. They were claimed to have health beneficial effects, such as lowering cholesterol. They occur in food as free sterols (FS), steryl esters (SE), and glycosylated conjugates comprised of steryl glucosides (SG) and acylated steryl glucosides (ASG).

Specific sterol profiles characteristic to certain plant families have been identified showing that a broad range ofminor sterols occurs as free sterols or glycosylated conjugates

This ion could be stanol (a sub-type of phytosterol with a saturated B-ring) derivative inferred from its C-ring fragmentation behaviour: 
\begin{itemize}
	\item higher intensities of m/z 149 than both 147 and 145
	\item higher intensities of m/z 161 than both 159 and 163
\end{itemize}
 


\section{Materials and methods}
\subsection{Chemicals}
Sitostanol standard (CAS Number 83-45-4, Avanti Polar Lipids Inc., USA) was transported and stored in -20 \degree C. Ethanol

\subsection{Apparatus}
UPLC-MS system (column C18, QTOF (VION, Waters, Milford, USA)

\subsection{UPLC-MS/MS analysis of Sitostanol}
%\cite{doi:10.1021/jf501509m}
%Mobile phase A: H2O
%Mobile phase B: Methanol
%Binary methods developed and optimized 
%Analysis methods referred\cite{sterolmsms}. Sitostanol stock solution (1 mg/mL) was prepared in 100\% ethanol. Further, stock solution was diluted by methanol, with the concentration of 0.02 mg/mL. 
%ESI positive mode was used to ionize.

\section{Retention Time (RT) and m/z in Different Matrix}
\begin{tabular}{|c|c|c|c|}
	\hline 
	Matrix & RT & m/z (ESI+) & Annotation \\ 
	\hline 
	Whole Grain Barley & 6.88 & 291.2683 & Unknown \\ 
	\hline 
	Urine & 6.71 & 291.2683 & Unknown \\ 
	\hline 
	Standard & 8.60 & 399.3989 & [Sitostanol-H2O+H]- \\ 
	\hline 
\end{tabular} 