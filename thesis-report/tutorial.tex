\documentclass[]{article}

%opening
\title{Tutorial: Introduction to LC-MS Based Metabolomics}
\author{Tu Hu\footnote{Data analyst, tu@nexs.ku.dk}}

\usepackage{makecell}
\begin{document}

\maketitle

\begin{abstract}
This tutorial uses simple and vivid language introducing basics of LC-MS based metabolomics to beginners.
\end{abstract}

\section{Background}
During my student life, I always like explaining difficult scientific questions in a simple, using analogies. 
In 2016, my girlfriend was notified . I tried to explain to her in the last air plane to 

\section{Principles of LC-MS}

\section{Terminologies and Explanations}


\section{Implementations and Feedbacks}
This tutorial has been used in following course

\begin{tabular}{|c|c|c|c|}
	\hline 
	No & \makecell{Number of \\participant(s)} & Course Title & Date \\ 
	\hline 
	1 & 1 & \makecell{Crash course on LC-MS Based \\ Metabolomics Data Preprocessing} & Apr/2019 \\ 
	\hline 
	2 & Unknown & \makecell{PhD course: Introduction\\ to Nutitional Metabolomics} & Jun/2019 \\ 
	\hline 
\end{tabular} 

\section{Reference}

\end{document}
