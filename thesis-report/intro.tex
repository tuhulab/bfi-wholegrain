%\subsection{R, Tidy Format and Data Wrangling}
%R is an open-source data analysis language.
%R is perfect for bioinformatics analysis.
%Tidy format is a term used in R describing a philosophy of storing data. Tidy format has the characteristics:
%\begin{itemize}
%	\item each variable is in its own column
%	\item each observation, or case, is in its own row
%\end{itemize}
%If data is stored in tidy form. Data scientists spend less time fighting with the tools and more time working on your analysis.

\subsection{Whole Grain Cereals and Their Health Beneficial Effects}
\subsection{Food Intake Biomarkers}
\subsection{LC-MS Based Metabolomics}
\subsection{Biostatistics Strategy in Metabolomics Research}
\subsubsection{Comparisions between Univariable and Multivariable statistics}
t-test has multiple testing problem. because when we do a t-test, normally we use a cutoff value of 0.05, it also means we take the risk of 5\% probability that it's NOT significantly different, but classified as different. this is called multiple testing problem. 

FALSE DISCOVERY problem in metabolomics.

how to overcome this problem? adjusted t-test, or reduce the cutoff to a reasnable value.

multivariable data analysis and univariable data analysis show different aspects of data. It is very common to observe analysis results are significant univariablely but not multivaribalely, also, it is common to see that another way. This means uni-/multi- variable data analysis both have their limitations. that's why it is recommanded that do both uni and multi variable data analysis for the same dataset.

However, how to integrate these analysis? are they chemically correlated? maybe one feature significant in univariable analysis is associated with another one in multivaribale data analysis? Maybe, one way is to first merge all these results together. in addition, because based on current technology limitation, it's impossible to identify OR intereprete all Metabolomics results, actually also time and resources. it actually exists priorities in identifying. better chance to identify, if they're correlated. meanwhile, if intensities are high.


\subsection{Identification}
\subsubsection{Level of Identification and communication confidence}
Reporting level of identification together with identification results can enhance communication confidence. Identification is recognized by far the most difficult part of Metabolomics research, especially concerning novel compound or biomarker discovery. 

In a single research project, not all structures or chemical information could be confirmed. Therefore, besides reporting chemical information (such as mass and structures), it is equally important to report the confidence of identification.

Five levels of confidence were proposed and applied extensively in xxx areas of LC-MS based compound identification\cite{?}.

\subsection{Validation of the Biomarker}

