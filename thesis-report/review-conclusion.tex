\acrshort{wg} barley attracted a lot of interest due to its health beneficial effects for preventing chronic disease. However, because barley is seldom consumed directly, currently there's no biomarkers reported for \acrshort{wg} barley intake both in human and animal studies. A lot of sparse information was reported from cereal and food chemistry could give hints for identification and validations of \acrshort{wg} barley's intake biomarkers.

Total \acrshort{ars} and their metabolites were reported to potentially indicate \acrshort{wg} wheat and rye intake. The homologues ratio of \acrshort{ars} C17:0/C21:0 was proposed to distinguish which whole grain type dominates in the diet. 